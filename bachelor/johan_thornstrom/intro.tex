%%% Intro.tex --- 
%% 
%% Filename: Intro.tex
%% Description: 
%% Author: Ola Leifler
%% Maintainer: 
%% Created: Thu Oct 14 12:54:47 2010 (CEST)
%% Version: $Id$
%% Version: 
%% Last-Updated: Thu May 19 14:12:31 2016 (+0200)
%%           By: Ola Leifler
%%     Update #: 5
%% URL: 
%% Keywords: 
%% Compatibility: 
%% 
%%%%%%%%%%%%%%%%%%%%%%%%%%%%%%%%%%%%%%%%%%%%%%%%%%%%%%%%%%%%%%%%%%%%%%
%% 
%%% Commentary: 
%% 
%% 
%% 
%%%%%%%%%%%%%%%%%%%%%%%%%%%%%%%%%%%%%%%%%%%%%%%%%%%%%%%%%%%%%%%%%%%%%%
%% 
%%% Change log:
%% 
%% 
%% RCS $Log$
%%%%%%%%%%%%%%%%%%%%%%%%%%%%%%%%%%%%%%%%%%%%%%%%%%%%%%%%%%%%%%%%%%%%%%
%% 
%%% Code:
\chapter{Introduktion}
\label{cha:johant_introduktion}
Mjukvaruutveckling med hjälp av ramverk är något som kan spara tid på ett projekt. När en person i ett projekt ska sätta sig in i nya mjukvarubibliotek och utveckla något med hjälp av de verktygen som finns tillgängliga kan tiden som personen kräver för att sätta sig in i strukturen ta en längre tid. Längre tid än vad projektets tidsramar tillåter. Tanken med ett ramverk är att packetera de olika bibliotek som kan används tillsammans. Det innebär att ramverket gör att utvecklaren inte behöver oroa sig över vissa delar av funktionalitet och integration. Det går naturligtvis också att inte använda sig av ett ramverk. I vissa fall är det mer lämpligt då utvecklaren får mer kontroll och frihet. Man kan utveckla en mer robust applikation än vad ett ramverk tillåter. Kostnaden kan dock vara att det tar mer tid utveckla om man inte är insatt i de bibliotek som används. 
\section{Syfte}
\label{sec:johant_syfte}
I ett projekt som varar under en kortare tid blir det viktigt att den tid som läggs åt implementeringen blir så pass effektiv som möjligt för att kunna producera en produkt i slutändan. När majoriteten av gruppmedlemmar saknar tidigare erfarenheter inom de programmeringsspråk som används blir tiden som läggs på upplärning extra viktig att den blir så pass kort och effektiv som möjligt. En projektgrupp med varierande tidigare erfarenheter av utveckling av webbapplikationer kan det finnas risk att det blir variation på hur mycket varje medlem i projektgruppen bidrar. Tanken är då att ett ramverk ska kunna bidra till att en projektmedlem med en bristande kunskap innan projektets start ska kunna komma upp i fas och bidra innan projektets tid tagit slut.
\section{Frågeställning}
\label{sec:frågeställning}
I det här projektet har vi i kandidatgrupp 2 i kursen TDDD96 använt oss av ramverket Meteor för utveckling av en webbapplikation. Målet med denna undersökning blir att undersöka ramverket Meteors förmåga att få igång utvecklarna så snabbt som möjligt. Det ska även undersökas om gruppmedlemmarnas tidigare erfarenheter om utveckling av webbapplikationer och hur deras erfarenheter återspeglas i ett ramverk vars grundprinciper bland annat är "simplicitet är lika med produktivitet". Frågeställningarna blir därmed.
\begin{enumerate}
\item Hur påverkar ramverket Meteor gruppmedlemmarnas inlärning av webbprogrammering
\item Vad för fördelar och nackdelar har ramverket Meteor för vår kandidatgrupp?
\end{enumerate}
\chapter{Bakgrund}
\label{cha:johant_bakgrund}
För webbprogrammerare är det viktigt att välja rätt JavaScript-ramverk som inte bara tjänar deras nuvarande webbprojektbehov utan också ger kod av hög kvalitet och bra prestanda. Omfattningen av detta arbete är att tillhandahålla en grundlig kvalitets- och prestandautvärdering av de mest populära JavaScript-ramarna, med beaktande av väl etablerade programvarukvalitetsfaktorer och prestandatester. Det stora resultatet är att vi lyfter fram fördelarna och nackdelarna med JavaScript-ramar inom olika områden av intresse och betecknar vilka och var de problematiska punkterna i deras kod, som troligen behöver förbättras i nästa versioner.
%\nocite{scigen}
%We have included Paper \ref{art:scigen}

%%%%%%%%%%%%%%%%%%%%%%%%%%%%%%%%%%%%%%%%%%%%%%%%%%%%%%%%%%%%%%%%%%%%%%
%%% Intro.tex ends here


%%% Local Variables: 
%%% mode: latex
%%% TeX-master: "demothesis"
%%% End: 
