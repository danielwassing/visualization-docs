\chapter{Slutsatser}
\label{cha:conclusion}

\section{Hur kan en webbapplikation för visualisering av kontinuerlig integration implementeras så att man skapar värde för kunden?}
När teknik och ramverk bestäms av kund är det största valet
som återstår inom projektet vilken arbetsmetodik som ska följas.
Med en agil utvecklingsmetod väljer gruppmedlemmar arbetsuppgifter
på friare hand än om en projektledare hade delat ut dem, vilket
lägger större ansvar på individen att faktiskt se till att få
rätt saker gjorda. Något som kan uppstå är att gruppmedlemmar t.ex.
försummar dokumentation för att programmera, eller på annat
sätt prioriterar fel saker ur projektets synpunkt. Att alla
är transparenta inom gruppen är viktigt så att man gemensamt
kan undvika sådana problem och kan styra projektet i rätt riktning.
\\ \\
Något som alltid varit rätt att fokusera på i projektet
är användarvänlighet och utseende. När en väldigt
gränssnitts-orienterad applikation utvecklas är det viktigt att
arbeta för att slutprodukten ska vara grafiskt
tilltalande och lätt att använda. Det kan vara subjektiva termer
så det är också viktigt att det är kundens perspektiv som betraktas, och då
kan pappersprototyper vara mycket användbara.

\section{Hur kan man använda pappersprototyper för att underlätta kravinsamling?}
När prototyperna visas för kund ges denne en möjlighet att ge återkoppling på en grafisk representation. Detta ger i sin tur gruppen möjlighet att får svar på vad som är accepterat och vad som bör förändras. Det specificeras genom att förtydliga och lägga till krav i kravspecifikationen. På så sätt drar kravinsamlingen nytta av prototypernas utformning och detaljeringsgrad. Att använda prototyper underlättar därför kravinsamlingen, samtidigt som utvecklingen gynnas av det. Det är alltså ett enkelt sätt att undersöka om tolkningen av kraven motsvarar kundens förväntningar.
\\ \\
Trots att pappersprototyper ger en mycket tydlig bild kan missförstånd
uppstå, och tekniska begränsningar kan upptäckas under projektets gång.
Att kontinuerligt demonstrera applikationen för kund ger möjligheten att
diskutera problem och designmissar och ser till att kunden är involverad i
arbetet och att slutprodukten överensstämmer med förväntningar.

\section{Vilket stöd kan man få genom att skapa och följa upp en systemanatomi?}
När kundens förväntningar etablerats behöver de brytas ned i faktiska implementationsdetaljer. Då kan en systemanatomi användas. En fördel med att göra en systemanatomi är att man får en indikation på hur den slutgiltiga produkten förväntas vara. Den har ett fokus på systemets funktionalitet och hur de relaterar till varandra. Projektgruppen får därmed en gemensam översikt inför implementering om hur de olika delarna i systemet håller ihop. Det ger däremot bara en grov översikt och kan inte ge någon djupare information om hur de olika delarna kommunicerar med varandra.

\section{Vilka erfarenheter kan dokumenteras från programvaruprojektet "Visualisering av kontinuerlig integration" som kan vara intressanta för framtida projekt?}
Det finns flera lärdomar efter projektet som kan föras vidare till framtida
arbeten, där de flesta är processrelaterade. Värdet av en tydlig
arbetsprocess kan ej underskattas, då det krävs för att alla ska veta
vad som ska göras och vilken riktning projektet förs i. Det möjliggör även
att införa t.ex. kodgranskning, regelbunden testning och iterativt arbete
med prototyper som del av processen.
\\ \\
Att arbeta med prototyper har varit värdefullt, men metoden som följdes
kunde ha formaliserats bättre. Att ha en tydlig process för när man skapar prototyper,
demonstrerar dem för kund och uppdaterar kravspecifikationen utifrån det
gör det möjligt att skapa arbetsuppgifter som mycket väl korresponderar mot kundens
önskemål och minimerar mängden missförstånd som annars kan uppstå vid kravinsamling.
\\ \\
Slutligen har vikten av att säkerställa gruppens kompetensen gjort sig tydlig. Om
kompetensen ej är tillräcklig kommer motivationen för att arbeta minska, så det
är viktigt att arbeta för allas tillgång till de resurser de behöver för att
lära sig.
