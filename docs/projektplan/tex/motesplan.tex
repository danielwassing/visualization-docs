\section{Mötesplan}
Utöver planerade veckomöten så hålls även extramöten vid behov. Under varje möte så skapar vi ett nytt mötesprotokoll utifrån vår protokoll-mall. Mötena sker internt i gruppen och i vissa fall även med handledare närvarande. Inför mötena ska gruppmedlemmar ha fört in punkter i dagordningen, detta för att få en bättre struktur på mötet när det väl är igång.
\subsection{Veckomöten}
Varje vecka så har vi ett oblikatoriskt veckomöte där vi går igenom vad vi har gjort under senaste veckan och vad vi ska göra under kommande vecka.
\subsection{Handledarmöten}
Handledarmötena sker tillsammans med handledare Lena Buffoni en gång per vecka. De mötena är en avstämning utifrån genomförande och planeringssynvinkel.\\ \\
Mötena kan ha två olika upplägg beroende på vad projektgruppen tycker känns relevant. Antingen håller projektgruppen ett eget möte med handledaren som observatör och kommer med kommentarer. Eller så håller hon i mötet utifrån mötespunkter som projektgruppen satt upp i förväg och meddelat handledaren.
\subsection{Kundmöten}
I projektets början används kundmötena för att reda ut projektets krav och vad applikationen förväntas innehålla. Senare i projektet används kundmötena för att stämma av hur utvecklingen går samt se att projektet går i rätt riktning. Kundmötena planeras in när kunden eller projektgruppen känner att behovet finns. Vid kundemötena representeras projektgruppen av teamledare och analysansvarig.