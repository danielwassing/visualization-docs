\section{Introduktion}
\subsection{Syfte}
Syftet med detta dokument är att etablera en standard för planering och utförande av tester för projektet. \\
\medskip
Dokumentet utgör en informell standard för utformning av testfall och formaliseras av testplanen.

\section{Utformning av testfall}
Testfallen formuleras som nedan;
\\
\\
\textbf{TestID:} \textit{Unik identifierare. Varje testfalls ID inleds av Tx där är ett löpnummer börjandes på 1.}\\
\textbf{Påstående:} \textit{Vad som ska verifieras. Se nedan (Formulering av testpåstående).}\\
\textbf{Antaganden:} \textit{De antaganden som understödjer testfallet.}\\
\textbf{Testdata: } \textit{Variabler och värden som testas.}\\
\textbf{Teststeg: } \textit{Arbetsgång för testet.}\\
\textbf{Förväntat resultat: } \\
\textbf{Uppnått resultat: } \textit{Vad resulterade testet i.}\\
\textit{Godkänt/Underkänt test:} \textit{Huruvida testet godkändes eller ej.}\\
\textbf{Kommentarer:} \\
\medskip
\\
\textit{Exempel:}\\
\\
\textbf{TestID:} T01\\
\textbf{Påstående:} Verifiera genom tidtagning med ramverkets tidtagninsfunktionalitet att hemsidan läses in på under två millisekunder. \\
\textbf{Antaganden:} Webbläsarens cacheminne är rensat. \\
\textbf{Testdata:} Uppmätt tid \\
\textbf{Teststeg:} Starta tidtagning, läs in hemsidan, läs av inläsningstiden. \\
\textbf{Förväntat resultat: } Hemsidans inläsningstid understiger två millisekunder. \\
\textbf{Uppnått resultat:} Hemsidan inläsningstid var en millisekund.\\
\textbf{Godkänd/Underkänt test: } GODKÄNT \\
\textbf{Kommentarer:} Testet är allmängiltigt för alla marknadsledande webbläsare. 

\subsection{Formulering av testpåstående}
För att formulera vad som ska verifieras används formatet; \\
\\
\textbf{Verifiera ...} \\
\textbf{... genom att använda ...} [verktyg, dialog] \\
\textbf{... med ...} [förutsättningar] \\
\textbf{... att ...} [vad som returneras, visas eller demonstreras]\\
\\
\textit{Exempel:}\\
Se testfallsexempel ovan.

\section{Enhetstestning}
\section{Systemtestning}
\section{Integrationstestning}