%%%%%%%%% lorem.tex --- 
%% 
%% Filename: lorem.tex
%% Description: 
%% Author: Ola Leifler
%% Maintainer: 
%% Created: Wed Nov 10 09:59:23 2010 (CET)
%% Version: $Id$
%% Version: 
%% Last-Updated: Wed Nov 10 09:59:47 2010 (CET)
%%           By: Ola Leifler
%%     Update #: 2
%% URL: 
%% Keywords: 
%% Compatibility: 
%% 
%%%%%%%%%%%%%%%%%%%%%%%%%%%%%%%%%%%%%%%%%%%%%%%%%%%%%%%%%%%%%%%%%%%%%%
%% 
%%% Commentary: 
%% 
%% 
%% 
%%%%%%%%%%%%%%%%%%%%%%%%%%%%%%%%%%%%%%%%%%%%%%%%%%%%%%%%%%%%%%%%%%%%%%
%% 
%%% Change log:
%% 
%% 
%% RCS $Log$
%%%%%%%%%%%%%%%%%%%%%%%%%%%%%%%%%%%%%%%%%%%%%%%%%%%%%%%%%%%%%%%%%%%%%%
%% 
%%% Code:

\chapter{Metod}
\label{cha:method}

In this chapter, the method is described in a way which shows how the
work was actually carried out. The description must be precise and
well thought through. Consider the scientific term
replicability. Replicability means that someone reading a scientific
report should be able to follow the method description and then carry
out the same study and check whether the results obtained are
similar. Achieving replicability is not always relevant, but precision
and clarity is.

Sometimes the work is separated into different parts, e.g.  pre-study,
implementation and evaluation. In such cases it is recommended that
the method chapter is structured accordingly with suitable named
sub-headings.

\section{Förstudie}
\label{sec:pre-study}

Retrospective är ett välanvänt verktyg bland de team som använder sig av Scrum och många har t.o.m. forskat
kring metoderna med agilutveckling. Därför har den här studien använt sig mycket av publicerade artiklar om agil arbetsmetodik
samt ett par böcker om hur man främst jobbar tillsammans i grupp. Samtliga artiklar och böcker går att finna bland referenserna.
Då den här rapporten är resultatet av dels en litterär studie och en praktisk studie så behövdes ett praktiskt sammanhang att undersöka.
Rapporten är en del av kursen TDDD96: 'Kandidatprojekt i programvaruutveckling' så sammanhanget blev det projekt som grupp 2 fick att genomföra.
Som nämnt i Introduktionen så Retrospective en del av Scrum och något som ska göras i slutet av varje iteration.
Då arbetsgruppen aldrig använt sig av Scrum som agil arbetsmetod så beslutades det att samtliga delar av Scrum skulle implementeras och därmed
även Retrospective-utvärderingar.

\section{Implementation}
\label{sec:implementation}

Längden av varje iteration under gruppens arbete sattes till 2 veckor långa och gav därmed utvärderingstillfälle med 2 veckors mellanrum.
Det planerades in ett 2 timmar långt tillfälle då samtliga av gruppens medlemmar kunde närvara vid slutet av iterationen och sedan genomfördes utvärderingen med ledning av team-ledaren. 

\subsubsection{Retrospective}
Den här delen ska förklara lite mer utförligt vad Retrospective är och vilka delar som ingår i en utvärdering av den typen.\\ \\
\textbf{Datainsamling}\\
Den första delen är datainsamlingen. Datainsamlingen i sig är uppdelad i 4 delar där den första delen är 'Teamutvärdering'. 
Här så får alla gruppmedlemmarna ett papper med ett flertal uttryck som de ska rangordna från 1-5 beroende på hur mycket de håller med uttrycket \cite{Team-utvärderings-pappret}. 
Den andra delen av datainsamlingen är 'Nöjdhet', där gruppens medlemmar ska svara på frågan 'Hur nöjda är vi med vårt arbete?'. Även här ska man svara enligt en 5-gradig skala.
\begin{itemize}
\item 5 = Jag tycker att vi är det bästa teamet på planeten. Vi jobbar bra ihop!
\item 4	= Jag är glad att jag är en del av teamet och nöjd med hur vårt team arbetar tillsammans
\item 3	= Jag är ganska nöjd. Vi jobbar bra tillsammans för det mesta
\item 2	= Jag har några stunder av tillfredsställelse, men inte tillräckligt
\item 1	= Jag är olycklig och missnöjd med vår nivå av lagarbete
\end{itemize}

Den tredje delen av datainsamlingen är 'Sammanställning', där man först identifierar de uttrycken eller områden som fått 1 eller 2 poäng av vardera gruppmedlem. Sedan sammanställer man gruppmedlemmarnas svar och fokuserar då främst på de som angett 1 eller 2 poäng.
Den sista delen av datainsamlingen är 'Bra \& Bättre', där man jobbar med frågorna.
\begin{itemize}
\item Vad är vi bra på?
\item Vad kan vi bli bättre på?
\end{itemize}
Man gör detta genom att varje gruppmedlem får en bunt post-it lappar. På lapparna så skriver varje medlem ner något som gruppen då är bra på eller kan bli bättre på med en åsikt per lapp. Efter att alla fått ihop några lappar var så går man sedan igenom lapparna genom att varje gruppmedlem läser upp en lapp i taget och sedan går man laget runt tills det inte längre finns några lappar kvar att läsa upp.

Nästa del av utvärderingen är 'Insikter'. Den här delen fungerar ungefär precis som 'Bra \& Bättre' från datainsamlingen, alltså på det sättet att varje gruppmedlem får en bunt post-it lappar att skriva på för att man sedan går igenom dessa lappar laget runt. I det här fallet ska man dock skriva ner vilka förbättringsförslag man har till de eventuella problem som identifierades under 'Bra \& Bättre'.

Efter att alla fått presentera sina förbättringsförslag så sammanställs dessa under nästa del som är 'Åtgärder'. Här bestämmer gruppen gemensamt vilka förbättringsförslag gruppen ska anamma och jobba med under nästa iteration av arbetet.

Den sista delen av utvärderingen är en kort summering och återkoppling på det som gåtts igenom. Här lyfts de saker som gruppen valde att fokusera på fram så att det blir tydligt vad som ska fokuseras på som förbättring under nästa iteration.
Man avslutar med att identifiera 3 saker som gruppen gjorde bra under retrospektivet.


%%%%%%%%%%%%%%%%%%%%%%%%%%%%%%%%%%%%%%%%%%%%%%%%%%%%%%%%%%%%%%%%%%%%%%
%%% lorem.tex ends here

%%% Local Variables: 
%%% mode: latex
%%% TeX-master: "demothesis"
%%% End: 
