%%% Intro.tex --- 
%% 
%% Filename: Intro.tex
%% Description: 
%% Author: Ola Leifler
%% Maintainer: 
%% Created: Thu Oct 14 12:54:47 2010 (CEST)
%% Version: $Id$
%% Version: 
%% Last-Updated: Thu May 19 14:12:31 2016 (+0200)
%%           By: Ola Leifler
%%     Update #: 5
%% URL: 
%% Keywords: 
%% Compatibility: 
%% 
%%%%%%%%%%%%%%%%%%%%%%%%%%%%%%%%%%%%%%%%%%%%%%%%%%%%%%%%%%%%%%%%%%%%%%
%% 
%%% Commentary: 
%% 
%% 
%% 
%%%%%%%%%%%%%%%%%%%%%%%%%%%%%%%%%%%%%%%%%%%%%%%%%%%%%%%%%%%%%%%%%%%%%%
%% 
%%% Change log:
%% 
%% 
%% RCS $Log$
%%%%%%%%%%%%%%%%%%%%%%%%%%%%%%%%%%%%%%%%%%%%%%%%%%%%%%%%%%%%%%%%%%%%%%
%% 
%%% Code:


\chapter{Introduktion}
\label{cha:introduction}

% http://agilemanifesto.org/
I de flesta projekt så krävs det verktyg för att kunna genomföra uppdraget. Verktyg kan också användas för att förbättra eller förenkla prcessen. Därför är val av verktyg och hur dessa ska konfigureras en viktig del av ett projektarbete.

%The introduction shall be divided into these sections:

\section{Motivation}
\label{sec:motivation}

% \cite{scigen}

% Behöver referens

När man utvecklar mjukvara är val av verktyg en viktig grund för att ett projekt ska flyta på som det ska. De verktyg som måste bestämmas är till exempel kommunikationsplatform och versionhantering. För att ge projektet så bra förutsättningar som möjligt så krävs välmotiverade val av verktyg.


%This is where the studied problem is described from a general
%point of view and put in a context which makes it clear that
%it is interesting and well worth studying. The aim is to make
%the reader interested in the work and create an urge to
%continue reading.

\section{Målsättning}
\label{sec:aim}

Målet med det här kappitlet är att berätta om vilka val av verktyg vi har gjort, hur det har gått att använda dessa verktyg och vilka ändraingar vi införde. Jag kommer även ta upp vad vi i efterhand anser kunnats göra bättre ller anorlunda.

%What is the underlying purpose of the thesis project?

\section{Frågeställningar}
\label{sec:research-questions}

Som relativt unga och inte helt erfarna så kan det vara svårt för oss som studenter att välja rätt verktyg, bland annat för att vi ej har provat dem alla. Det är därför viktigt att vi utverderar våra val för att i framtiden kunna ha bättre förutsättningar för att välja rätt verktyg. De frågeställningar som jag tänker svara på är:

%This is where the research questions are described.
%Formulate these as explicit questions, terminated with a
%question mark. A report will usually contain several different
%research questions that are somehow thematically connected.
%There are usually 2-4 questions in total.

%Examples of common types of research questions (simplified
%and generalized):

\begin{enumerate}
\item Vilka verktyg valde vi och varför?
\item Vilka ändringar eller konfigurationer gjorde vi och varför?
\item Vilka ändringar i val av verktyg och hur de är konfigurerade skulle vi vilja ha gjort om vi fick gå tillbaka i tiden?
\end{enumerate}


%Observe that a very specific research question almost always
%leads to a better thesis report than a general research question
%(it is simply much more difficult to make something good
%from a general research question.)

%The best way to achieve a really good and specific research
%question is to conduct a thorough literature review and get
%familiarized with related research and practice. This leads to
%ideas and terminology which allows one to express oneself
%with precision and also have something valuable to say in the
%discussion chapter. And once a detailed research question
%has been specified, it is much easier to establish a suitable
%method and thus carry out the actual thesis work much faster
%than when starting with a fairly general research question. In
%the end, it usually pays off to spend some extra time in the
%beginning working on the literature review. The thesis
%supervisor can be of assistance in deciding when the research
%question is sufficiently specific and well-grounded in related
%research.

\section{Avgränsningar}
\label{sec:delimitations}

Studien kommer endast baseras på kandidatarbetet som utförs av
grupp 2 i kursen TDDD96 våren 2017.

%This is where the main delimitations are described. For
%example, this could be that one has focused the study on a
%specific application domain or target user group. In the
%normal case, the delimitations need not be justified.

%\nocite{scigen}
%We have included Paper \ref{art:scigen}

%%%%%%%%%%%%%%%%%%%%%%%%%%%%%%%%%%%%%%%%%%%%%%%%%%%%%%%%%%%%%%%%%%%%%%
%%% Intro.tex ends here


%%% Local Variables: 
%%% mode: latex
%%% TeX-master: "demothesis"
%%% End: 
