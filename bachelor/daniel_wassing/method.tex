%%% lorem.tex --- 
%% 
%% Filename: lorem.tex
%% Description: 
%% Author: Ola Leifler
%% Maintainer: 
%% Created: Wed Nov 10 09:59:23 2010 (CET)
%% Version: $Id$
%% Version: 
%% Last-Updated: Wed Nov 10 09:59:47 2010 (CET)
%%           By: Ola Leifler
%%     Update #: 2
%% URL: 
%% Keywords: 
%% Compatibility: 
%% 
%%%%%%%%%%%%%%%%%%%%%%%%%%%%%%%%%%%%%%%%%%%%%%%%%%%%%%%%%%%%%%%%%%%%%%
%% 
%%% Commentary: 
%% 
%% 
%% 
%%%%%%%%%%%%%%%%%%%%%%%%%%%%%%%%%%%%%%%%%%%%%%%%%%%%%%%%%%%%%%%%%%%%%%
%% 
%%% Change log:
%% 
%% 
%% RCS $Log$
%%%%%%%%%%%%%%%%%%%%%%%%%%%%%%%%%%%%%%%%%%%%%%%%%%%%%%%%%%%%%%%%%%%%%%
%% 
%%% Code:

\chapter{Metod}
\label{cha:wassing-method}
För att få svar på de frågor som togs upp i avsnitt \ref{sec:wassing-research-questions} så intervjuades ett antal projektmedlemmar ur olika grupper under iteration 3 och 4 i kursen. Deltagarna valdes ut slumpmässigt beroende på vilka som var tillgängliga och samarbetsvilliga. En träff bestämdes för varje deltagare. På träffen hölls en intervju där deltagaren fick svara på frågor. Alla deltagare fick svara på samma frågor. Frågorna hade i förväg utarbetats för att tangera frågeställningarna men även generera öppna svar, då det ansågs vara av intresse att få reda på hur olika grupper hade tacklat olika problem på vägen. Frågorna som ställdes återges nedan: %möjligen ändra till appendix senare
\begin{itemize}
\item{Med vilka verktyg dokumenterar din grupp?}
\item{Hur togs besluten av val av verktyg för dokumentation?}
\item{Vilka fördelar med dessa verktyg ser du/ni?}
\item{Vilka nackdelar ser du/ni?}
\item{Hade ni kunnat föreställa er att använda andra verktyg?}
\item{Med tanke på att kandidatarbetet behandlar väldigt mycket dokumentation så kan det vara bra att ha mallar att utgå ifrån (inte minst för slutrapporten). Det finns mallar för kandidatuppsatser tillgängliga. Har dessa mallar varit användbara för er? Har de varit applicerbara?}
\end{itemize}
Det som sades på intervjuerna antecknades och renskrevs för tydlighet. Svaren har sammanställts utefter vilka verktyg som deltagarnas grupper valt och tas upp i avsnitt \ref{cha:wassing-results}.
\\ \\
Till grund för diskussionen som presenteras i avsnitt \ref{cha:wassing-discussion} ligger även konferensbidraget \textit{Documentation in Systems Development: A Significant Criterion for Project
Success} \cite{docsystemdev}. %kanske mer här sen men det är fuckat att hitta vettiga källor

%In this chapter, the method is described in a way which shows how the
%work was actually carried out. The description must be precise and
%well thought through. Consider the scientific term
%replicability. Replicability means that someone reading a scientific
%report should be able to follow the method description and then carry
%out the same study and check whether the results obtained are
%similar. Achieving replicability is not always relevant, but precision
%and clarity is.

%Sometimes the work is separated into different parts, e.g.  pre-study,
%implementation and evaluation. In such cases it is recommended that
%the method chapter is structured accordingly with suitable named
%sub-headings.

%%%%%%%%%%%%%%%%%%%%%%%%%%%%%%%%%%%%%%%%%%%%%%%%%%%%%%%%%%%%%%%%%%%%%%
%%% lorem.tex ends here

%%% Local Variables: 
%%% mode: latex
%%% TeX-master: "demothesis"
%%% End: 
