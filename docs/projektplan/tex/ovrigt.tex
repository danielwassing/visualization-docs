\section*{Appendix A - Vår anpassning av Scrum}
Nedan följer en definition av den Scrumversion som gruppen arbetar utifrån.

\subsection*{Anpassning av Scrum}
Det här dokumentet syftat till att specificera de anpassningar som gruppen gör jämfört med “renodlad” Scrum. Det är skrivet innan utvecklingen börjat och kommer tjäna som utgångspunkt för framtida ändringar och förbättringar.
\subsection*{Sprinter}
I scrum är arbetet strukturerat i två-fyra veckor långa sprinter. I vår tillämpning kommer vi använda oss av två veckor långa sprinter.
\subsection*{Scrummöten}
I scrum håller scrummästaren dagliga scrummöten där teamet går igenom vad de gjort sedan gårdagen, vad de skall åstadkomma under dagen och vilka problem som förutses. På grund av att gruppen ej arbetar på heltid med projektet och har olika kurser parallellt är schemakrockar en verklighet och möten kommer hållas veckovis istället för dagligen.
\subsection*{Produktägaren}
I scrum agerar produktägaren som representant för kunden och det är dennes uppgift att leverera prioriterade user stories till utvecklingsteamet. Det kräver stor kunskap hos en enskild person för att kunna ta fram korrekta user stories och prioritera dem rätt, vilket inte kan förväntas av en grupp som prövar scrum för första gången utan någon utbildning i det. På grund av det så är det hela gruppens uppgift att under mötestid arbeta fram user stories och prioritera dem korrekt. Gruppens analysansvarig kommer även att hålla i produktägarens alla kommunikativa bitar med kund.
\subsection*{Utvecklingsteamet}
Vårt utvecklingsteam består av åtta personer, vilket ligger inom Scrums rekommenderade antal på sex-nio. Det är självorganiserande och kommer organiseras av scrummästaren endast i nödfall.
\subsection*{Scrummästaren}
Gruppens scrummästare kommer agera likt Scrums definition. Denne kommer främst arbeta för att undanröja hinder för teamets utvecklingsarbete och organisera scrummöten.
\\
Källa: http://www.scrumguides.org/\cite{website:scrum_guide} 2016 update

