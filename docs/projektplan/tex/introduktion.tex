\section{Introduktion}
Nedan beskrivs bakgrunden till projektets utförande och syftet med detta.

\subsection{Syfte}
Syftet med detta dokument är att definiera projektgruppens tillvägagångssätt för att leverera produkten med tillhörande dokumentation. Projektplanen tydliggör projektets mål och vilka avgränsningar och resurser som är givna.

\subsection{Bakgrund}
Projektet skapades på begäran av Kristian Sandahl och Ola Leifler vid Linköpings Universitet, som en del i kursen TDDD96: Kandidatprojekt i programvaruutveckling. Kristian är engagerad i det nationella konsortiet Software Center som är ett samarbete mellan flera svenska universitet och företag. Software Center har formulerat ett behov att inom industrin överskådligt kunna se hur olika processer och händelser beror av varandra för att bland annat ta strategiska beslut i utvecklingsprocessen för CI.

\subsection{Definitioner}
Tydliggörande av, för vissa, okända termer som används i dokumentet.
\begin{description}[leftmargin=!,labelwidth=\widthof{\bfseries <Google Drive>}]
\item [\bf Google Drive\cite{website:googledrive}] Molnbaserad lagringstjänst.
\item [\bf Scrum\cite{website:scrum}] En agil utvecklingsmetod.
\item [\bf Slack\cite{website:slack}] Molnbaserad kommunikationsplattform för projektgrupper.
\item [\bf Sprint\cite{website:sprint}] En iterationscykel i Scrum.
\item [\bf JavaScript\cite{website:javascript}] Ett programmeringsspråk för webutveckling.
\item [\bf Eiffel\cite{website:eiffel}] Ett ramverk för datahantering, utvecklat av Ericsson.
\item [\bf LaTeX\cite{website:latex}] Ett scriptspråk för att producera dokument av hög kvalitet.
\item [\bf Git\cite{website:git}] Ett versionshanteringsprogram för hantering av parallell programmering.
\end{description}
