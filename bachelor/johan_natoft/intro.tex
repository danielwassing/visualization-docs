%%% Intro.tex --- 
%% 
%% Filename: Intro.tex
%% Description: 
%% Author: Ola Leifler
%% Maintainer: 
%% Created: Thu Oct 14 12:54:47 2010 (CEST)
%% Version: $Id$
%% Version: 
%% Last-Updated: Thu May 19 14:12:31 2016 (+0200)
%%           By: Ola Leifler
%%     Update #: 5
%% URL: 
%% Keywords: 
%% Compatibility: 
%% 
%%%%%%%%%%%%%%%%%%%%%%%%%%%%%%%%%%%%%%%%%%%%%%%%%%%%%%%%%%%%%%%%%%%%%%
%% 
%%% Commentary: 
%% 
%% 
%% 
%%%%%%%%%%%%%%%%%%%%%%%%%%%%%%%%%%%%%%%%%%%%%%%%%%%%%%%%%%%%%%%%%%%%%%
%% 
%%% Change log:
%% 
%% 
%% RCS $Log$
%%%%%%%%%%%%%%%%%%%%%%%%%%%%%%%%%%%%%%%%%%%%%%%%%%%%%%%%%%%%%%%%%%%%%%
%% 
%%% Code:


\chapter{Introduktion}
\label{cha:introduction}

Mjukvaruutveckling idag är oftast en laginsats och man pratar mycket om att jobba i grupper eller såkallade team.
Att jobba tillsammans är en vetenskap i sig och det krävs en del från samtliga medlemmar i en grupp för att arbetet ska fungera bra.
Ett av sätten som olika team brukar jobba på är agilt, vilket i sin tur har flera olika metoder som man kan följa för att jobba just agilt.
En av de här metoderna är Scrum, en agil arbetsmetod som används av många team runtom i världen.
Scrum liksom de flesta agila arbetsmetoderna karaktäriseras av tidsbegränsade iterationer och möjligheten att kunna hantera förändring i
projektet på ett effektivt och dynamiskt sätt. Efter varje iteration så ska varje team, enligt Scrums metodik, utvärdera teamets arbete och hur teamet arbetar tillsammans i ett så kallat \textit{Retrospective}. Denna rapport kommer att prata mer om denna utvärderingsmetod och om hur den påverkar ett teams effektivitet samt hur beslut som tas efter en sådan utvärdering förstärks på grund av just dessa utvärderingar.


%The introduction shall be divided into these sections:

\section{Motivation}
\label{sec:motivation}

% \cite{scigen}



Med en mjukvaruindustri som skiftar alltmer mot agila utvecklingsmetoder
är det viktigt för studenter som studerar datateknik att redan
under sin utbildning introduceras till ett agilt klimat för att
vara väl förberedda inför arbetslivet. Med flera kurser som läses
samtidigt kan det dock vara svårt att tillämpa Scrum i sin rena form
så en mängd kompromisser och anpassningar behöver göras för att
frigöra tid för övriga studier.


%This is where the studied problem is described from a general
%point of view and put in a context which makes it clear that
%it is interesting and well worth studying. The aim is to make
%the reader interested in the work and create an urge to
%continue reading.

\section{Målsättning}
\label{sec:aim}

Målet med denna rapport är att ta reda på hur Retrospective som utvärderingsverktyg
påverkar arbetsgruppens sätt att arbeta tillsammans på ett effektivt sätt. 

%What is the underlying purpose of the thesis project?

\section{Frågeställningar}
\label{sec:research-questions}

Att prata om hur arbetet man utför som en grupp kan effektiviseras är en sak men kan bli en
helt annan när det kommer till att faktiskt implementera eventuella förändringar.
Vissa förändringar behöver inte alltid vara särskilt konkreta utan kan även innefatta t.ex. hur man 
kommunicerar med varandra inom arbetsgruppen. Dessa förändringar är dock minst lika viktiga för hur
väl arbetet inom gruppen kommer att fungera och därför är rapportens frågeställning:

%This is where the research questions are described.
%Formulate these as explicit questions, terminated with a
%question mark. A report will usually contain several different
%research questions that are somehow thematically connected.
%There are usually 2-4 questions in total.

%Examples of common types of research questions (simplified
%and generalized):

\begin{enumerate}
\item Till vilken grad hjälper Retrospektiv arbetsgruppen att fullfölja beslut som effektiviserar gruppens arbete?
\end{enumerate}


%Observe that a very specific research question almost always
%leads to a better thesis report than a general research question
%(it is simply much more difficult to make something good
%from a general research question.)

%The best way to achieve a really good and specific research
%question is to conduct a thorough literature review and get
%familiarized with related research and practice. This leads to
%ideas and terminology which allows one to express oneself
%with precision and also have something valuable to say in the
%discussion chapter. And once a detailed research question
%has been specified, it is much easier to establish a suitable
%method and thus carry out the actual thesis work much faster
%than when starting with a fairly general research question. In
%the end, it usually pays off to spend some extra time in the
%beginning working on the literature review. The thesis
%supervisor can be of assistance in deciding when the research
%question is sufficiently specific and well-grounded in related
%research.

\section{Avgränsningar}
\label{sec:delimitations}

Den praktiska studien kommer endast baseras på kandidatarbetet som utförs av
grupp 2 i kursen TDDD96 våren 2017.

%This is where the main delimitations are described. For
%example, this could be that one has focused the study on a
%specific application domain or target user group. In the
%normal case, the delimitations need not be justified.

%\nocite{scigen}
%We have included Paper \ref{art:scigen}

%%%%%%%%%%%%%%%%%%%%%%%%%%%%%%%%%%%%%%%%%%%%%%%%%%%%%%%%%%%%%%%%%%%%%%
%%% Intro.tex ends here


%%% Local Variables: 
%%% mode: latex
%%% TeX-master: "demothesis"
%%% End: 
