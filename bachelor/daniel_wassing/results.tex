%%% lorem.tex --- 
%% 
%% Filename: lorem.tex
%% Description: 
%% Author: Ola Leifler
%% Maintainer: 
%% Created: Wed Nov 10 09:59:23 2010 (CET)
%% Version: $Id$
%% Version: 
%% Last-Updated: Wed Nov 10 09:59:47 2010 (CET)
%%           By: Ola Leifler
%%     Update #: 2
%% URL: 
%% Keywords: 
%% Compatibility: 
%% 
%%%%%%%%%%%%%%%%%%%%%%%%%%%%%%%%%%%%%%%%%%%%%%%%%%%%%%%%%%%%%%%%%%%%%%
%% 
%%% Commentary: 
%% 
%% 
%% 
%%%%%%%%%%%%%%%%%%%%%%%%%%%%%%%%%%%%%%%%%%%%%%%%%%%%%%%%%%%%%%%%%%%%%%
%% 
%%% Change log:
%% 
%% 
%% RCS $Log$
%%%%%%%%%%%%%%%%%%%%%%%%%%%%%%%%%%%%%%%%%%%%%%%%%%%%%%%%%%%%%%%%%%%%%%
%% 
%%% Code:

\chapter{Resultat}
\label{cha:wassing-results}
I det här kapitlet presenteras resultaten från intervjuerna som hållits med deltagare i kursen TDDD96, våren 2017. För intervjuernas struktur, se avsnitt \ref{cha:wassing-method}.

\section{Latex}
\label{sec:wassing-latex}
De flesta av de tillfrågade deltagarnas grupper hade använt sig av typsättningsspråket LaTeX. LaTeX upplevdes ha en \textit{snöbollseffekt}, dvs. att det kunde vara trögt att komma igång, men takten att producera nya och bättre dokument ökade ju längre projektet framskred. Trenden var att bara en minoritet någonsin hade skrivit i LaTeX i varje grupp, om någon hade gjort det över huvud taget. Det handlade ofta om bara en eller två personer som var bekanta med det sedan tidigare. Det var nästan alltid en person med tidigare kunskap i LaTeX som blev utsedd att vara dokumentansvarig.

\subsection{Fördelar med LaTeX}
Fördelarna var många, menade de som använt det. Programmerare har ofta en tendens att inte vilja använda muspekaren och bara knappa på tangentbordet. Detta är något som LaTeX tillgodoser då det av ren märkspråksnatur handlar om att skriva kod och text. Grafiska representationer för infogande av bilder och dylikt existerar inte, utan allt löses genom ren kodskrivning. LaTeX ger möjligheten att göra stora övergripande ändringar av enorma dokument med ändring av bara några enstaka rader kod och transfererar full kontroll till skribenten.

\subsection{Nackdelar med LaTeX}
Den uppenbara nackdelen som nämndes var inlärningskurvan. LaTeX har väldigt mycket möjligheter, men som vilket annat programmeringsspråk som helst så krävdes det en tid innan medlemmarna var införstådda med syntax som krävdes för producering av dokument. Latex är ett typsättningssystem och inte ett ordbehandlingssytem, vilket leder till att det man skriver inte direkt reflekterar det man får i slutändan\footnote{Detta kallas på engelska \textit{WYSIWYG}, What you see is what you get}. Då tidsbrist ofta var ett problem så blev inlärningskurvan en ganska central punkt att ta ställning till vid valet av dokumentationsverktyg. Mer om det i avsnitt \ref{cha:wassing-discussion}. 

\section{Git}
\label{sec:wassing-gitresults}
Alla tillfrågade grupper hade använt sig av Git, dock inte alltid för dokumentationen. Versionshanteringssystem såsom Git är centralt för utveckling av mjukvara, men är inte lika erkänt för dokumentation. För- och nackdelar som redovisas nedan är sett helt ur ett dokumenteringsperspektiv, ej utveckling.
\\ \\
En stor del av de tillfrågade förklarade att de använde Git vid dokumentation utöver utveckling. Det visade sig vara användbart att kunna versionshantera även dokument då individer kan råka radera eller ändra på fel ställe i dokument.

\subsection{Fördelar med Git}
De som använde Git för dokumentation citerade versionshanteringens egenskaper: Git tillåter användare att hålla reda på historik och förändringar i dokument, samt gör det möjligt att backa tillbaka i tiden eller att spåra vem som har gjort vad. Detta var till nytta för samtliga som versionshanterade sina dokument med Git. Se avsnitt \ref{sec:wassing-git} för information om hur versionshantering går till.

\subsection{Nackdelar med Git}
Två nackdelar nämndes, dessa var inlärningskurvan och \textit{merge konflikter} \footnote{Mer känt som \textit{merge conflicts} på engelska}.
\\ \\
Versionshantering har som det mesta annat en inlärningskurva. För att komma igång med Git så tillhandahölls interna guider och utbildning från ansvariga personer i grupper.
\\ \\
Merge konflikter är när två eller flera personer utför ändringar på samma ställe i ett dokument och alla som vill uppdatera dokumentet efter den första personen får en konflikt från Git. Dokument som fick konflikter behövde lösas manuellt och omaktualiseras. För en oerfaren person kunde mergekonflikter ses som något jobbigt då personen inte visste vad som skulle stå kvar och tas bort.

\section{Google Drive}
\label{sec:wassing-googledrive}
Alla tillfrågade förklarade att deras grupper hade gemensamma Drive-mappar där dokumentation sparas. Utsträckningen av vilken dokumentation som sparades varierade, vissa valde att dokumentera allt i Google Drive, medans andra nöjde sig med gruppdokument som inte låg i kunds intresse att se. Sådana dokument var t.ex. riktlinjer för hur gruppmedlemmarna gjorde olika saker, mötesprotokoll och rapporter.

\subsection{Fördelar med Google Drive}
\label{subsec:wassing-googledrive-pros}
Fördelarna var många, menade tillfrågade. Drive exekveras i webbläsaren och är därför plattformsoberoende. Tillgång till internet var inget problem på campus eller vid arbete hemifrån så det ansågs inte vara en begränsning.\\
Systemet är fokuserat på ordbehandling, vilket gör att det som skrivs är det som fås i slutändan (WYSIWYG). Google Drive ansågs vara pedagogiskt och det fanns en stor supportdatabas som enkelt kunde hittas genom sökning med godtycklig sökmotor om frågor uppstod. Drive tillät dessutom flera människor att jobba samtidigt på ett och samma dokument. Stöd för offline-arbete fanns om internet inte fanns att tillgå, men det funkade dock bara om dokumentet redan fanns cache:at i webbläsaren.

\subsection{Nackdelar med Google Drive}
Det nämndes inte mycket nackdelar med Google Drive förutom att det krävde internet. Dock togs det upp att Google Drive hade mindre funktionalitet för skapande av högklassiga dokument än vad exempelvis Microsoft Office hade. Office hade mer användarvänlighet, ett enkelt användargränssnitt och en stor mängd mallar att använda vid skapande av dokument som Google Drive saknade.

\section{ShareLaTeX}
\label{sec:wassing-sharelatex}
ShareLaTeX kunde sammanfattas som Google Drive fast för LaTeX. Trots att ShareLaTeX är en mix av två frekvent använda verktyg så var det få som svarade att de använde sig av det. Det beror på att det krävs ett premiumkonto för att kunna arbeta mer än 2 personer samtidigt på ett dokument. Premiumkonto går att uppnå på 2 olika sätt, antingen genom att betala för det eller genom att använda ShareLaTeXs inbjudningssystem\footnote{Mer känt som Referral-system på engelska} till att bjuda in ett antal andra personer att börja använda ShareLaTeX. %finns även alternativet att alla sitter på samma konto

\subsection{Fördelar med ShareLaTeX}
ShareLaTeX behåller LaTeXs syfte, men har möjligheten att låta flera personer arbeta på samma dokument samtidigt, precis som Google Drive. Detta minskade inlärningskurvan då personer lärde sig från parprogrammering där någon part alltid satt och skrev kod. Tillfrågade personer pekade även på att det fanns en enorm support för olika bibliotek och plugins i ShareLaTeX som inte behövde hämtas hem lokalt. ShareLaTeX krävde heller ingen editor att jobba med då allt skedde direkt i webbläsaren, vilket sparade ännu mer tid.
 
\subsection{Nackdelar med ShareLaTeX}
Versionshanteringen ansågs vara bristfällig, enligt tillfrågade som använt ShareLaTeX. Om flera arbetade på samma dokument kunde det dessutom ge krockar i testbarheten då kompileringen kunde vara icke-genomförbar om någon annan arbetade i ett annat stycke i samma dokument.

\section{Microsoft Office}
\label{sec:wassing-microsoft-office}
Ingen av de tillfrågade personernas grupper använde sig av Microsoft Office paketet för teknisk dokumentation. Det tas dock ändå upp här då vissa aspekter av användande har förekommit.

\subsection{Word}
En av deltagarna i intervjun poängterade att enstaka personer i sin grupp skrev temporära dokument och anteckningar i Word innan en dokumentationsstandard slogs fast. Detta byttes ganska snart ut mot Google Drive.

\subsection{PowerPoint}
Några tillfrågade svarade att PowerPoint användes för att skapa presentationer till seminarietillfällen i kursen. Detta framför allt för att de kände sig bekväma med att använda alla funktioner som PowerPoint hade att erbjuda. En person pekade även på att PowerPoint kan göra en hel del mer och har utökad funktionalitet för design av slides, animeringar och dylikt än exempelvis \textit{Google Presentations} \footnote{Google Presentations är en del av Google Drive}.
%http://www.presentation-guru.com/google-slides-versus-powerpoint/

\subsection{Excel}
En central del av dokumenteringen var tidsrapporteringen. En tillfrågad person demonstrerade sin grupps tidsrapportering, där de använde ett excel-ark som databas för rapporteringen. Excel-arket hade fått en mängd funktionalitet inlagt, som att kunna plocka ut exakt data för vissa tidsperioder i klart format. Detta hade varit till hjälp för teamledaren när denne skulle rapportera in till handledaren vad gruppen gjort under en specifik vecka.

\subsection{Fördelar med Office-paketet}
Beslutet att använda sig av Office-paketet grundade sig alltid i tidsåtgången att producera färdiga dokument, enligt tillfrågade. Antingen för att de snabbt ville slänga ihop någon presentation fort med mjukvara vars avancerade funktioner de kände till, eller för att de hade tidigare kunskaper om hur mjukvaran kunde användas till att utföra uppgifter som behövde utföras under kursens gång, exempelvis tidsrapportering.

\subsection{Nackdelar med Office-paketet}
Anledningen varför Office inte användes var enligt tillfrågade främst avsaknaden av två nyckelfunktioner. Dessa var versionshantering och möjligheten att arbeta flera på samma dokument. De två sakerna gemensamt gjorde att vidare efterforskningar för att hitta lösningar till problemen (såsom plugins för att köra binära filer med Git) inte gjordes då det ansågs vara ett slöseri med tid. Det ansågs som en mindre tidsinvestering att lära gruppen att använda beprövade verktyg i stället.

%This chapter presents the results. Note that the results are presented
%factually, striving for objectivity as far as possible.  The results
%shall not be analyzed, discussed or evaluated.  This is left for the
%discussion chapter.

%In case the method chapter has been divided into subheadings such as
%pre-study, implementation and evaluation, the result chapter should
%have the same sub-headings. This gives a clear structure and makes the
%chapter easier to write.

%In case results are presented from a process (e.g. an implementation
%process), the main decisions made during the process must be clearly
%presented and justified. Normally, alternative attempts, etc, have
%already been described in the theory chapter, making it possible to
%refer to it as part of the justification.

%%%%%%%%%%%%%%%%%%%%%%%%%%%%%%%%%%%%%%%%%%%%%%%%%%%%%%%%%%%%%%%%%%%%%%
%%% lorem.tex ends here

%%% Local Variables: 
%%% mode: latex
%%% TeX-master: "demothesis"
%%% End: 
