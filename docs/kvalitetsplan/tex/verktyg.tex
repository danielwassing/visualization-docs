\section{Verktyg, tekniker och metoder}
Projektgruppen har beslutat om standarder för arbetsverktyg, tekniker och metoder för att underlätta samarbetet. 

\subsection{Trello}
Vi kommer att använda oss av verktyget Trello\cite{website:trello} för att organisera och dela upp arbete i gruppen. 
\\ \\
Varje feature, bugfix och dokumentation blir en issue i Trello som kommer ha en etikett för vilken typ av problem det är, samt en etikett för vilket krav från kravspecifikationen som berörs.

\subsection{GitLab}
Vi använder GitLab\cite{website:gitlab} i samarbete med Linköpings Universitet som värd för båda våra git-repositoryn. I GitLab har vi möjlighet att helt använda git och får dessutom tillgång till några extra funktioner som Slack-integration. 
\\ \\
Källkoden kommer förvaras och versionhanteras på GitLab. Detta betyder att alla kommer jobba mot GitLab när vi delar och versionhanterar den kod vi skriver. Alla branches och merge request ska hänvisa till en eller flera issues i Trello, detta görs genom att koppiera in en länk som pekar på en specifik issue i Trello.
\\ \\
GitLab ska inte användas för att fördela arbete eller för att föra diskussioner om någonting.

\subsection{Google Drive}
Vi använder Google Drive\cite{website:googledrive} för att dela interna dokumenet som enkelt modifieras av alla samtidigt. 
\\ \\
Under möten har sekreteraren mötets dagordning på Google Drive öppet för att lätt följa de anteckningar som skrivs eller refereras till. Google Drive-dokument används även till att dela upp och organisera dokumentskrivandet innan själva praktiska projektet startar, som till exempel denna text.
\\ \\
Google Drive ska inte användas för de ändamål som Trello och GitLab används, utan kompletterar och underlättar för när vi delar interna dokument som inte ska publiceras på något sätt.

