\section{Fasplan}
Projektet utförs i iterationer enligt vår scrum-modell (se stycke 6.6, scrum). Varje iteration är 2 veckor lång.

\subsection{Förstudie}
Under förstudien skall projektplan, kravspecifikation, kvalitetsplan, statusrapport och systemanatomi skrivas. Dokument som ska vara påbörjade men inte behöver vara färdiga är arkitekturdokument och testplan.

\subsection{Iterationer}
Under iterationerna kommer vi jobba med prio 1 kraven tills alla är uppfyllda. Om vi har tid kvar när alla krav av prioritet 1 är uppfyllda så kommer vi fortsätta arbeta med prioritet-2-kraven.

%Iteration 3 är den sista övergripande fasen av projektet. När kunden anser att kraven har blivit tillräckligt uppfyllda anses projektet avklarat. Därefter skall systemet levereras tillsammans med teknisk dokumentation och användarhandledning. Slutlig dokumentation ska skrivas och revideras. En slutrapport skall skrivas. All dokumentation skall lämnas in.

%Under iteration 2 så skall projektet genomföras. Det ska konstrueras och testas enligt kravspecifikationen, kvalitetsplanen, arkitekturdokumentet och testplanen. Det skall även skrivas teknisk dokumentation och en användarhandledning ska påbörjas.