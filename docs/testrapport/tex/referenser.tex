\section{Avvikelser}
Testningsprocessen har avvikit från det testplanen på följande punkter; 

\begin{itemize}
\item \textbf{Användbarhetstest} - ingen användbarhetstestning har varit planerad för den passerade iterationen. Arbete med systemets användbarhet har utförts genom utvecklingsprocessen, men specifika användbarhetstester har inte utförts enligt spårbara och vetenskapliga metoder. Användharbetstest innefattar förväntansuppfyllnad och lättmanöverbarhet vilka bägge lämnas till nästa iteration när produkten antagit en mer levererbar form. 
\item \textbf{Prestandatest} - prestandatester har inte uttryckligen utförts även om utvecklingsarbetet har skett med hänsyn till olika prestandakriterier.
\item \textbf{Lasttest} - systemet har utvecklats med prestandakraven i beaktning, och då systemet har bedömts kunna hantera det antal händelser som återfinns i kravspecifikationen (\textit{Kravspecifikation - Prestandakrav}) så har detta test förkastats.
\item \textbf{Stresstest} - efter en upptäckt minnesläcka i serverprogramvaran utfördes ett oplanerat stresstest som ett led i att verifiera buggfixen, se utvecklingskortet \textit{https://trello.com/c/Dx5tlyym} på Trello. Då inga andra stresstester finns planerade anses detta avvika från det planerade testarbetet. 
\item \textbf{GUI-testning} - se användbarhetstest.
\end{itemize}
\medskip
Observera att då inga av dessa testpunkter har uttryckligen planerats iterationsvis så anses även de testpunkter som inte utförts avvika från testplanen.\\
\\
Utformning av testfall har också avvikit från det planerade, där fältet \textbf{TestID} har utelämnats från de testfall formulerade i Trello-korten eftersom testfallen kan refereas till med utvecklingskortens titel eller dess delningslänk.