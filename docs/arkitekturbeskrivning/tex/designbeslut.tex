\section{Designbeslut}
I det här kapitlet tas de beslut upp som har påverkat arkitekturen.
\subsection{Byte av kodbas}
Ett förslag om att byta kodbas för serversidan till Java lades fram till kunden. Detta för att kunna tydliggöra separation av klient och server. Efter kundens begäran gjordes valet att fortsätta med det givna ramverket Meteor.
\subsection{Meteors applikationstruktur}
För att applikationen ska vara läsbar ska Meteors standard för strukturering av applikationer följas.\cite{website:meteorstructure} Detta för att det ska finnas en tydlig mall som alla i projektet följer, därmed blir det lättare för resten av gruppen att sätta sig in i kod de själva inte har skrivit. Det fyller dessutom den önskan kunden hade om att applikationen ska kunna vidareutvecklas. Genom att ha en ha en dokumenterad standard uppfylls kundens önskan om vidareutvecklingsbarhet.
\subsection{Kodstandard för JavaScript}
Applikationen följer JavaScipt implementationen av ECMAScript 6 standarden.\cite{website:ecmascript} ECMAScript är ett standardiserat scriptspråk. Det var utvecklat för att standardisera JavaScript för främja olika oberoende implementationer av scriptdialekter. Meteors kodstandard rekommenderar det och efter gruppens beslut bestämdes det att följa den.