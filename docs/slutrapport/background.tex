
\chapter{Bakgrund}
\label{cha:background}
Även om KI-system implementeras för att hantera komplexitet, kan även de
bli komplexa och svåröverskådliga. Konsortiet Software Center arbetar tillsammans med
bland annat Linköpings Universitet för att studera storskaliga mjukvarumiljöer
för kontinuerlig integration och det var de som lade grunden för det här projektet.
\\ \\ 
Kunden för detta projekt var Kristian Sandahl och Ola Leifler ifrån Institutionen
för datateknik (IDA) på Linköpings Universitet som på Software Centers vägnar ville ha
en webbapplikation för att visualisera KI-system. Målet var att ge ett beslutsstöd
för intressenter utan teknisk kompetens genom att på ett lättöverskådligt
sätt kunna identifiera flaskhalsar och eventuella problem i KI-system.
\\ \\
Projektet utfördes som en del av kursen TDDD96 Kandidatprojekt i programvaruutveckling
av åtta studenter från civilingenjörsprogram i datateknik och mjukvaruteknik.
Tillsammans med projektdirektivet tillhandahölls även en tidigare utvecklad
prototyp av systemet.

%I samarbete med konsortiet Software center, med 10 företag och 5
%universitet, arbetar vi med att studera storskaliga miljöer för kontinuerlig integration (CI).
%Idag har dessa miljöer vuxit i komplexitet samtidigt som möjligheterna för att gå ännu
%längre – kontinuerlig leverans – blir lockande för företagen.
