\section{Antaganden och beroenden}
Avsnittet beskriver antaganden och beroenden som systemet kommer att anpassas efter.

\begin{table}[h!]
  \centering
  \caption{En tabell över mjukvara.}
  \def\arraystretch{1.5}
  \begin{adjustbox}{max width=\textwidth}
    \begin{tabularx}{\textwidth}{ | l | l | X | }
      \hline
      \textbf{Mjukvara} & \textbf{Version} & \textbf{Beskrivning} \\
      \hline
      Meteor & 1.4.2.6 & Ett ramverk för utveckling av webbapplikationer i JavaScript. Kunden gav projektgruppen en kodbas skriven i Meteor. \\ 
      \hline
      Node.js & 3.10.10 & En exekveringsmiljö för en server i JavaScript. Meteors server använder sig av Node.js.\\
      \hline
      Cytoscape.js & 3.5.0 & Ett JavaScript-bibliotek för rendering av grafer. Används istället för att själva implementera renderingen och sparar därmed tid.  \\
      \hline
      Dagre.js & 0.7.4 & Ett JavaScript-bibliotek för layout av grafer. Förenklar processen att rita ut riktade grafer. \\
      \hline
      Vis.js & 4.19.1 & Ett JavaScript-bibliotek för visualisering. Ger en dynamisk tidslinje för den aggregerade grafen och en tabell för tabelläge. \\
      \hline
    \end{tabularx}
  \end{adjustbox}
  \label{tab:mjukvara}
\end{table}
\


\begin{table}[h!]
  \centering
  \caption{En tabell över exekveringsmiljöer.}
  \def\arraystretch{1.5}
  \begin{adjustbox}{max width=\textwidth}
    \begin{tabularx}{\textwidth}{ | l | X | }
      \hline
      \textbf{Operativsystem} & \textbf{Webbläsare} \\
      \hline
      Ubuntu 16.04 & Firefox 51.0.1 \\
      \hline
      macOS Sierra 10.12.3 & Chrome 55.0.2883.95 \\
      \hline
      Windows 8.1, 10 & Chrome 55.0.2883.95 \\
      \hline
    \end{tabularx}
  \end{adjustbox}
  \label{tab:exekveringsmiljoer}
\end{table} 
\ \\
Tabell \ref{tab:mjukvara} beskriver den mjukvara som systemet kommer att baseras på. Tabell \ref{tab:exekveringsmiljoer} innehåller versionsnummer av exekveringsmiljöer för applikationens klientsida. De exekveringsmiljöerna har valts för de är de miljöer som projektgruppen använder. 