 \chapter{\textit{Retrospektiv} - hur det hjälper team att effektivisera sitt arbete av Johan Nåtoft}

%%%%%%%%%%%%%%%% INTRODUKTION %%%%%%%%%%%%%%%%

\section{Introduktion}
\label{cha:johan_n-introduction}
Mjukvaruutveckling idag är oftast en laginsats och man pratar mycket om att jobba i grupper eller så kallade team.
Att jobba tillsammans är en vetenskap i sig och det krävs en del arbete från samtliga medlemmar i en grupp för att det gemensamma arbetet ska fungera bra.
Ett av sätten som olika team brukar jobba på är agilt, vilket i sin tur har flera olika metoder som man kan följa för att jobba just agilt.
En av de här metoderna är Scrum, en agil arbetsmetod som används av många team runtom i världen.
Scrum liksom de flesta agila arbetsmetoderna karaktäriseras av tidsbegränsade iterationer och möjligheten att kunna hantera förändring i
projektet på ett effektivt och dynamiskt sätt. Efter varje iteration så ska varje team, enligt Scrums metodik, utvärdera teamets arbete och hur teamet arbetar tillsammans i ett så kallat \textit{Retrospektiv}. Denna rapport kommer att prata mer om denna utvärderingsmetod och om hur den påverkar ett teams effektivitet. Den tar även upp hur beslut som tas i och med en sådan utvärdering förstärks på grund av denna utvärderingsmetod.

\subsection{Motivering}
\label{sec:johan_n-motivation}
Retrospektiv är en välanvänd metod för utvärderingar inom Scrum. Det finns dock de som prioriterar bort dessa utvärderingar då de inte känner att de får ut något av dom. Att analysera Retrospektiv som utvärderingsmetod i ett applicerbart sammanhang och hur mycket vinning som faktiskt kan fås ur att genomföra dessa kan motivera andra att ge det ett ordentligt försök. Retrospektiv är till för att kontinuerligt förbättra hur en grupp människor arbetar tillsammans, så att inte ge metoden ett försök kan ses som en ovilja att samarbeta överhuvudtaget.


\subsection{Målsättning}
\label{sec:johan_n-aim}
Målet med denna rapport är att ta reda på hur Retrospektiv som utvärderingsverktyg
påverkar arbetsgruppens sätt att arbeta tillsammans på ett effektivt sätt. 


\subsection{Frågeställningar}
\label{sec:johan_n-research-questions}

Att prata om hur arbetet man utför som en grupp kan effektiviseras är en sak men kan bli en
helt annan när det kommer till att faktiskt implementera eventuella förändringar.
Vissa förändringar behöver inte alltid vara särskilt konkreta utan kan även innefatta t.ex. hur man 
kommunicerar med varandra inom arbetsgruppen. Dessa förändringar är dock minst lika viktiga för hur
väl arbetet inom gruppen kommer att fungera och därför är rapportens frågeställning:

%Examples of common types of research questions (simplified
%and generalized):
\begin{itemize}
\item Till vilken grad hjälper Retrospektiv arbetsgruppen att fullfölja beslut som effektiviserar gruppens arbete samt förbättrar gruppens samarbete?
\end{itemize}


\subsection{Avgränsningar}
\label{sec:johan_n-delimitations}

Den praktiska studien kommer endast baseras på kandidatarbetet som utförs av
grupp 2 i kursen TDDD96 våren 2017. Detta kan begränsa resultatet då en studie av flera grupper hade kunnat ge annorlunda data.


%%%%%%%%%%%%%%%% TEORI %%%%%%%%%%%%%%%%

\section{Teori}
\label{cha:johan_n-theory}

Retrospektiv är ett välanvänt verktyg bland de team som använder sig av Scrum och många har forskat
kring metoderna med agil utveckling. Därför har den här studien använt sig mycket av publicerade artiklar om agil arbetsmetodik
samt ett par böcker om hur man främst jobbar tillsammans i grupp. De artiklar och böcker som denna studie använt sig av går att finna bland referenserna. I denna del presenteras utvärderingsverktyget \textit{Retrospektiv} utförligare och det ges en klar uppfattning om hur man använder sig av det. Det presenteras även insikter ifrån diverse artiklar som har behandlat ämnen som involverar Retrospektiv och grupparbete inom agil utveckling.

\subsection{Retrospektiv}
Den här delen förklarar lite mer utförligt vad Retrospektiv är och vilka delar som ingår i en utvärdering av den typen. De olika stegen är hämtade från boken \textit{Agile retrospectives: Making good teams great}.\cite{agile_retrospectives} (s.19)\\\\
\textbf{Datainsamling}\\
Den första delen är datainsamlingen. Datainsamlingen i sig är uppdelad i fyra delar där den första delen är \textbf{Teamutvärdering}. 
Här så får alla gruppmedlemmarna ett papper med ett flertal uttryck som de ska rangordna från 1-5 beroende på hur mycket de håller med uttrycket.\ref{cha:team_utvardering} Den andra delen av datainsamlingen är \textbf{Nöjdhet}, där gruppens medlemmar ska svara på frågan \textit{'Hur nöjda är vi med vårt arbete?'}. Även här ska man svara enligt en 5-gradig skala.
\begin{itemize}
\item 5 = Jag tycker att vi är det bästa teamet på planeten. Vi jobbar bra ihop!
\item 4	= Jag är glad att jag är en del av teamet och nöjd med hur vårt team arbetar tillsammans
\item 3	= Jag är ganska nöjd. Vi jobbar bra tillsammans för det mesta
\item 2	= Jag har några stunder av tillfredsställelse, men inte tillräckligt
\item 1	= Jag är olycklig och missnöjd med vår nivå av lagarbete
\end{itemize}

Den tredje delen av datainsamlingen är \textbf{Sammanställning}, där man först identifierar de uttrycken eller områden som fått ett eller två poäng av vardera gruppmedlem. Sedan sammanställer man gruppmedlemmarnas svar och fokuserar då främst på de som angett 1 eller 2 poäng.
Den sista delen av datainsamlingen är \textbf{Bra \& Bättre}, där man jobbar med frågorna.
\begin{itemize}
\item Vad är vi bra på?
\item Vad kan vi bli bättre på?
\end{itemize}
Man gör detta genom att varje gruppmedlem får en bunt post-it lappar. På lapparna så skriver varje medlem ner något som gruppen då är bra på eller kan bli bättre på med en åsikt per lapp. Efter att alla fått ihop några lappar var så går man sedan igenom lapparna genom att varje gruppmedlem läser upp en lapp i taget och sedan går man laget runt tills det inte längre finns några lappar kvar att läsa upp.\\\\
\textbf{Insikter}\\
Nästa del av utvärderingen är \textbf{Insikter}. Den här delen fungerar ungefär som \textbf{Bra \& Bättre} från datainsamlingen, alltså på det sättet att varje gruppmedlem får en bunt post-it lappar att skriva på för att man sedan går igenom dessa lappar laget runt. I det här fallet ska man dock skriva ner vilka förbättringsförslag man har till de eventuella problem som identifierades under \textbf{Bra \& Bättre}.\\\\
\textbf{Åtgärder}\\
Efter att alla fått presentera sina förbättringsförslag så sammanställs dessa under nästa del som är \textbf{Åtgärder}. Här bestämmer gruppen gemensamt vilka förbättringsförslag gruppen ska anamma och jobba med under nästa iteration av arbetet.\\\\
\textbf{Summering och avslut}\\
Den sista delen av utvärderingen är en kort summering och återkoppling på det som gåtts igenom. Här lyfts de saker som gruppen valde att fokusera på fram så att det blir tydligt vad som ska fokuseras på som förbättring under nästa iteration.
Man avslutar med att identifiera tre saker som gruppen gjorde bra under retrospektivet.

\subsection{Erfarenheter från andra studier}

Det finns många artiklar och även böcker om hur man genomför ett Retrospektiv på bästa sätt. Många av dessa pratar dock ofta om väldigt grundläggande principer och tips för att genomföra ett bra Retrospektiv, inte något revolutionerande att tillägga om genomförandena egentligen.
Det finns dock insikter från männsikor som arbetat med Retrospektiv som ett verktyg i sina team och de som använt sig av verktyget på ett, vad som anses vara, korrekt vis har funnit att de som team utvecklats enormt.\cite{wow_team} Av de team som tagit till sig de agila principerna helhjärtat så har dessa visat stor förbättring inom sättet de arbetar på och även inom sättet de arbetar tillsammans.\cite{wow_team} 

%The main purpose of this chapter is to make it obvious for
%the reader that the report authors have made an effort to read
%up on related research and other information of relevance for
%the research questions. It is a question of trust. Can I as a
%reader rely on what the authors are saying? If it is obvious
%that the authors know the topic area well and clearly present
%their lessons learned, it raises the perceived quality of the
%entire report.

%After having read the theory chapter it shall be obvious for
%the reader that the research questions are both well
%formulated and relevant.

%The chapter must contain theory of use for the intended
%study, both in terms of technique and method. If a final thesis
%project is about the development of a new search engine for
%a certain application domain, the theory must bring up related
%work on search algorithms and related techniques, but also
%methods for evaluating search engines, including
%performance measures such as precision, accuracy and
%recall.

%The chapter shall be structured thematically, not per author.
%A good approach to making a review of scientific literature
%is to use \emph{Google Scholar} (which also has the useful function
%\emph{Cite}). By iterating between searching for articles and reading
%abstracts to find new terms to guide further searches, it is
%fairly straight forward to locate good and relevant
%information, such as \cite{test}.

%Having found a relevant article one can use the function for
%viewing other articles that have cited this particular article,
%and also go through the article’s own reference list. Among
%these articles on can often find other interesting articles and%
%thus proceed further.

%It can also be a good idea to consider which sources seem
%most relevant for the problem area at hand. Are there any
%special conference or journal that often occurs one can search
%in more detail in lists of published articles from these venues
%in particular. One can also search for the web sites of
%important authors and investigate what they have published
%in general.

%This chapter is called either \emph{Theory, Related Work}, or
%\emph{Related Research}. Check with your supervisor.%


%%%%%%%%%%%%%%%% METOD %%%%%%%%%%%%%%%%

\section{Metod}
\label{cha:johan_n-method}

Då den här rapporten är resultatet av dels en litterär studie men även en praktisk studie så behövdes ett praktiskt sammanhang att undersöka.
Rapporten är en del av kursen TDDD96: 'Kandidatprojekt i programvaruutveckling' så sammanhanget blev det projekt som grupp 2 fick att genomföra.
Som nämnt i introduktionen så är Retrospektiv en del av Scrum och något som ska göras i slutet av varje iteration.
Då arbetsgruppen aldrig använt sig av Scrum som agil arbetsmetod men nu valt att testa på det så beslutades det att samtliga delar av Scrum skulle implementeras och därmed även Retrospektiv-utvärderingar.

\subsection{Implementation}
\label{sec:johan_n-implementation}

Längden av varje iteration under gruppens arbete sattes till två veckor och gav därmed utvärderingstillfälle med två veckors mellanrum. Vid det första tillfället planerades det in ett två timmar långt tillfälle då samtliga av gruppens medlemmar kunde närvara vid slutet av iterationen och sedan genomfördes utvärderingen med ledning av team-ledaren. Vid det första tillfället som en sådan här utvärdering hölls så förklarades syftet med genomförandet och de olika momenten presenterades i tur och ordning. Gruppen accepterade förutsättningarna för utvärderingen och sedan genomfördes momentet på ett övertydligt och pedagogiskt vis, åtminstone som avsikt. Kommande Retrospektiv utfördes på ett mer effektivt sätt där gruppen snabbare kunde färdigställa datainsamlingen samt komma till beslut om åtgärder inför nästa iteration. Diskussionerna som uppstod i samband med \textit{Sammanställningen} utvecklades även till det bättre allteftersom Retrospektiven fortskred. All data som framställdes under dessa tillfällen samlades in av team-ledaren efter sessionen och analyserades senare. Detta inkluderade själva teamutvärderings-formulären som samtliga medlemmar fyllde i och även de små lapparna med förbättringsförlag och hyllningar som gruppen producerat.\ref{cha:team_utvardering} 

\subsection{Utvärdering av Retrospektiven}
\label{sec:johan_n-evaluation-of-retrospectives}

Efter att samtliga Retrospektiv genomförts så skickades det ut ett formulär till arbetsgruppen där medlemmarna fick utvärdera Retrospektiv som utvärderingsverktyg och hur det har påverkat gruppens arbete.\ref{cha:retrospektiv_eval_form} Där ställdes 5 frågor till medlemmarna, vilket var följande:
\begin{itemize}
\item Visste du vad Retrospektiv var innan projektets start och hade du provat på det?
\item Hur användbart har du känt att Retrospektiv varit för gruppens arbete och samarbete?
\item Finns det någon specifik del av Retrospektiv som du tycker har hjälpt gruppen extra mycket?
\item Har gruppen tagit till sig de beslut som tagits på Retrospektiv-mötena och därmed förbättrat sig?
\item Finns det något kring Retrospektiv-metoden som kunde ha gjorts bättre?
\end{itemize}

%%%%%%%%%%%%%%%% RESULTAT %%%%%%%%%%%%%%%%

\section{Resultat}
\label{cha:johan_n-results}

Denna del presenterar dels resultat i form av en sammanställning av den data som samlades in efter respektive Retrospektiv samt resultat som genererades från utvärderingsformuläret där medlemmarna fick utvärdera Retrospektiv som utvärderingsmetod.

\subsection{Sammanställning av data}
\label{sec:johan_n-res-data}
Här presenteras en sammanställning av all data som samlats in i och med retrospektiven. Sammanställningen har utförts genom att räkna ut en snittsiffra för var och en av uttrycken per tema i Team-utvärderingsformuläret som svarades på under retrospektiven.\ref{cha:team_utvardering} Denna siffra har sedan utvecklats med tiden allteftersom retrospektiven utförst och redovisas i de horisontellt intilliggande kolumnerna namngivna efter datum i sammanställnings-dokumentet.\ref{cha:sammanstallning_av_data} De celler till höger om tematiteln för varje del innehåller en sammanfattning som skrevs i samband med retrospektiven och som representerade de delar som vissa i gruppen tyckte att gruppen borde jobba på.\\ I kolumnen \textbf{Tolkning av resultat} så har utvecklingen analyserats kort för att ge understöd till diskussionen. I botten av sammanställningen redovisas antalet områden som enligt gruppen antingen förbättrades, försämrades eller inte hade någon förändring.\\På bladet 'Bra \& Bättre' visas en sammanställning av de punkter som gruppens medlemmar tyckte att gruppen antingen var bra på eller kunde bli bättre på. Plustecknena ('+') följande vardera punkt i cellerna representerar antelet medlemmar som hade liknande om inte samma åsikt.


\subsection{Utvärderingsformuläret}
\label{sec:johan_n-res-form}
Här presenteras resultaten från utvärderingsformuläret för varje fråga där resultaten vid möjlighet har sammanställts.\ref{cha:retrospektiv_eval_form} Vid sådana möjligheter så har svaren upplevts som väldigt lika om inte identiska.
\begin{itemize}
\item Visste du vad Retrospektiv var innan projektets start och hade du provat på det?
\end{itemize}
Endast två av projektgruppens medlemmar visste vad Retrospektiv var innan projektets början. Dock hade ingen av dessa två testat på metoden.

\begin{itemize}
\item Hur användbart har du känt att Retrospektiv varit för gruppens arbete och samarbete?
\end{itemize}
Denna fråga ställdes med ett förväntat svar på en skala från 1-10.
Resultatet från samtliga medlemmar blev ett snitt på 8,1 med en median på 8 och inga svar lägre än 7.

\begin{itemize}
\item Finns det någon specifik del av Retrospektiv som du tycker har hjälpt gruppen extra mycket?
\end{itemize}
Här svarade gruppen väldigt lika förutom några avvikelser. Det fanns en tydlig enighet inom gruppen om att sessionerna hade varit en bra plats att lyfta problem de kände inom gruppen och att det gav folk chansen att klaga. Ett svar framförde värdet av att kontinuerligt applicera nya förbättringar inom gruppen genom projektets gång. 

\begin{itemize}
\item Har gruppen tagit till sig de beslut som tagits på Retrospektiv-mötena och därmed förbättrat sig?\\
Denna fråga hade fyra svarsalternativ, vilket var:
\begin{itemize}
\item Ja, till 100\%
\item Till stor del åtminstone
\item Inte direkt, men den är väl lite bättre
\item Nej, ingen förbättring har gjorts
\end{itemize}
7 av 8 gruppmedlemmar svarade \textit{"Till stor del åtminstone"} medans en medlem svarade \textit{"Inte direkt, men den är väl lite bättre"}.
\end{itemize}


\begin{itemize}
\item Finns det något kring Retrospektiv-metoden som kunde ha gjorts bättre?\\
Tre gruppmedlemmar valde att svara på denna fråga då denna var den enda frågan som inte var obligatorisk i formuläret.
De förslag på förbättring som gavs var:
\begin{itemize}
\item Applicera en teknik för att få med samtliga medlemmar i diskussionerna.
\item Nöjdhetsskalan kunde varit mellan 1-10 istället för 1-5. Sen fanns det en osäkerhet om hurvida denna skala bör vara inkluderad överhuvudtaget.
\item Det bör ha funnits uppföljning på tidigare Retrospektiv.
\end{itemize}
\end{itemize}

%%%%%%%%%%%%%%%% DISKUSSION %%%%%%%%%%%%%%%%

\section{Diskussion}
\label{cha:johan_n-discussion}
I denna del kommer metoden och resultatet att diskuteras. Resultatet kommer att diskuteras utifrån de underrubriker som presenterades där samt analyseras med stöd av den litterära studie som gjordes i teori-kapitlet.\ref{cha:johan_n-theory}

%This chapter contains the following sub-headings.

\subsection{Metod}
\label{sec:johan_n-discussion-method}
När det kommer tillgenomförandet av retrospektiven så är tillvägagångssättet inte något som uppvisar några större svårigheter. Det var första gången som någon i projektgruppen genomfört ett retrospektiv och det märktes att retrospektiv som utvärderingsmetod är något som man behöver genomföra ett par gånger för att en grupp ska bli bekväm med det. Det som kan vara svårt är att gruppmedlemmarna måste vara bekväma med att öppna upp sig för varandra. Detta är ett kriterie för god transparans och diskussion, vilket leder till klargöranden av de potentiella problem som gruppen kan ha. Om medlemmarna är bekanta och även bekväma med att öppna upp sig för varandra så underlättar detta förbättringsarbetet inom gruppen då många problem kan grunda sig i bristande kommunikation. \\ \\
Då gruppen schemalade retrospektiven i början av iterationen men aldrig innan iterationsplaneringen kan detta varit en bidragande faktor till att visst förbättringsarbete kunde ta mer plats. Hade insikterna och besluten som togs på retrospektivet funnits på bordet då iterationen planerades kan arbete ha sett annorlunda ut.
Faktumet att gruppens rollfördelning inte riktigt motsvarar den hos en projektgrupp inom näringslivet bidrar nog till stor del för hur olika arbetsgrupper lyckas implementera förändringar på arbetsplatser. Som de beskriver i \textit{Agile Retrospectives: Making Good Teams Great} så underlättar förändring om den kan understödjas av en styrande faktor, såsom en chef eller liknande.\cite{agile_retrospectives} Då denna projektgrupp består av en samling studenter så är det svårt att efterlikna den styrande makt som kan finnas inom näringslivet hos en chef eller annan ledare. 

\subsection{Resultat}
\label{sec:johan_n-discussion-results}
Denna del kommer att diskutera de två delarna av resultatet separat och sedan avsluta med en övergripande analys.

\subsubsection{Sammanställning av data}
\label{sec:johan_n-discussion-results-data}
Sammanställningen visar att 11 av 26 områden, eller påståenden, har förbättrats över tiden som gruppen har genomfört retrospektiven. Att retrospektiven ska ha en förbättrande effekt på gruppens samarbete kunde vi se bl.a. från artikeln \textit{From team to wow-team}.\cite{wow_team} Deras grupp såg till att genomföra ordentliga retrospektiv som en konsekvens av striktare implementering av en agil arbetsmetodik och därmed förbättra deras samarbete avsevärt. 11 av 26 områden är dock inte ens hälften vilket kan ifrågasätta retrospektiv som metod för förbättring. \\\\
Som Esther Derby och Diana Larsen skriver i \textit{Agile Retrospectives: Making Good Teams Great} att även om retrospektiv är en bra metod för att belysa en grupps styrkor och svagheter så måste det även finnas en metod för att genomföra förändringen.\cite{agile_retrospectives}(sid. 142) Den aktuella projektgruppen har haft en vilja att förändras och idéer för förbättring. Det som har saknats är dock tydliga metoder för att genomföra dessa förbättringar. Först och främst måste de förbättringar som man bestämts sig för ha stöd från gruppen. Detta gäller främst i början av dess implementation då detta är den perioden då det även är lättast att glömma bort de saker som ska leda till en förbättring. Hälften av de områden som utvärderades har inte nått någon signifikant förändring eller varierat minimalt. Detta kan också vara bekräftande för faktumet att gruppen inte haft någon klar metod för att implementera sina förbättringar. Då ingen metod finns att tillgå är det väldigt svårt att förändra gruppens beteende.\\\\
De två områden som sett en försämring är hur gruppen berömde de individuella bedrifterna och hur individuella medlemmar tog övergripande ansvar. Att gruppen inte berömde varandra kan kopplas till hur samhörigheten verkar brustit i någon större utveckling. Angående det övergripande ansvaret från gruppens medlemmar kan det endast spekuleras att något sådant möjligtvis bottnar i medlemmarnas skilda ambitionsnivåer. I delen 'Bra \& Bättre' kan man utläsa hur gruppen har haft en vilja att bli bättre på att uppmuntra och berömma varandra.\ref{cha:sammanstallning_av_data} Man kan även se hur gruppen har efterfrågat fler gruppaktiviteter för att stärka sammanhållningen. Då man kan anta att den bristande sammanhållningen varit bidragande till den dåliga vanan att inte berömma individuella bidrag, så förstärks även här antagandet att det existerat brister i själva utförandet av potentiella förbättringar av gruppen samarbete. Vad som däremot motsäger detta är hur vissa medlemmar tydligt påpekat under 'Bra \& Bättre' i det sista genomförda retrospektivet att gruppen blivit bättre på att uppskatta andra medlemmars arbete. Detta visar på hur olika uppfattningar gruppens medlemmar har haft om den visade uppskattningen inom gruppen. Man kan fråga sig om uppskattningen endast riktats mot vissa specifika medlemmar i gruppen och om även detta kan ha något att göra med sammanhållningen i gruppen. \\\\
I 'Bra \& Bättre' kan vi även se tydliga förbättringar som gruppen har lyckats genomföra. Så som bl.a. hur möten effektiviserats mellan iterationer samt hur gruppen sett ett behov av att sitta tillsammans under arbetstid och sedan genomfört denna förbättring. Att sitta tillsammans kan vara en av de förbättringar som gynnat uppvisandet av uppskattning inom gruppen då individuella bidrag lättare kan berömmas om gruppen är samlad på ett och samma ställe.


\subsubsection{Formulär}
\label{sec:johan_n-discussion-results-form}
Även om 75\% av gruppen inte kände till retrospektiv innan projektets början så såg definitivt 100\% av gruppen nyttan i att använda det. Att många ansåg att retrospektiven var ett bra tillfälle att lyfta problem inom gruppen vittnar om att ett sådant forum var behövligt. Att arbeta i grupp är sällan enkelt, och man behöver ge gruppen möjlighet att utvecklas när den behöver det. När gruppen blev tillfrågad hur väl implementeringen av bestämda förbättringar gått så svarade de flesta att gruppen tagit till sig det mesta, men inte allt. Även här visas bristen på metodik och näring till de beslut som gruppen tog i och med retrospektiven för att förbättra sitt arbete och samarbete. Även om det bara brister till en mindre del så kan det finnas en chans att de förbättringar som faktiskt lyckades leta sig in i gruppen gjorde detta genom ren viljestyrka istället för en genomtänkt implementation. 

%Are there anything in the results that stand out and need be
%%analyzed and commented on? How do the results relate to the
%material covered in the theory chapter? What does the theory
%imply about the meaning of the results? For example, what
%does it mean that a certain system got a certain numeric value
%in a usability evaluation; how good or bad is it? Is there
%something in the results that is unexpected based on the
%literature review, or is everything as one would theoretically
%expect?


%This is where the applied method is discussed and criticized.
%Taking a self-critical stance to the method used is an
%important part of the scientific approach.

%A study is rarely perfect. There are almost always things one
%could have done differently if the study could be repeated or
%with extra resources. Go through the most important
%limitations with your method and discuss potential
%consequences for the results. Connect back to the method
%theory presented in the theory chapter. Refer explicitly to
%relevant sources.

%The discussion shall also demonstrate an awareness of methodological
%concepts such as replicability, reliability, and validity. The concept
%of replicability has already been discussed in the Method chapter
%(\ref{cha:method}). Reliability is a term for whether one can expect
%to get the same results if a study is repeated with the same method. A
%study with a high degree of reliability has a large probability of
%leading to similar results if repeated. The concept of validity is,
%somewhat simplified, concerned with whether a performed measurement
%actually measures what one thinks is being measured. A study with a
%high degree of validity thus has a high level of credibility. A
%discussion of these concepts must be transferred to the actual context
%of the study.

%The method discussion shall also contain a paragraph of
%source criticism. This is where the authors’ point of view on
%the use and selection of sources is described.

%In certain contexts it may be the case that the most relevant
%information for the study is not to be found in scientific
%literature but rather with individual software developers and
%open source projects. It must then be clearly stated that
%efforts have been made to gain access to this information,
%e.g. by direct communication with developers and/or through
%discussion forums, etc. Efforts must also be made to indicate
%the lack of relevant research literature. The precise manner
%of such investigations must be clearly specified in a method
%section. The paragraph on source criticism must critically
%discuss these approaches.

%Usually however, there are always relevant related research.
%If not about the actual research questions, there is certainly
%important information about the domain under study.

%There must be a section discussing ethical and societal
%aspects related to the work. This is important for the authors
%to demonstrate a professional maturity and also for achieving
%the education goals. If the work, for some reason, completely
%lacks a connection to ethical or societal aspects this must be
%explicitly stated and justified in the section Delimitations in
%the introduction chapter.

%In the discussion chapter, one must explicitly refer to sources
%relevant to the discussion.



\section{Slutsatser}
\label{cha:johan_n-conclusion}
Retrospektiv är ett väldigt bra sätt för projektgrupper att synliggöra brister och problem inom gruppen. Om medlemmarna i gruppen är öppna mot varandra så blir det lättare att identifiera källan till problemen. Med färre problem inom gruppen så ökar även chansen att sammanhållningen blir bättre. Om gruppen känner en starkare sammanhållning så kommer samarbetet att gynnas vilket i sin tur kan öka produktiviteten. För att lösa problem inom gruppen eller förbättra vissa aspekter av samarbetet så krävs tydliga tillvägagångssätt. Likt som alla förändringar så kommer dessa att ta tid och därför behövs det näring och tålamod för att gruppen ska ta till sig dom. Om gruppen inte jobbar tillsammans och aktivt för att förbättringarna ska implementeras ordentligt kan det resultera i att vissa förbättringar glöms bort. \\
\subsection{Till vilken grad hjälper då Retrospektiv arbetsgruppen att fullfölja beslut som effektiviserar gruppens arbete samt förbättrar gruppens samarbete?}
Retrospektiven i sig hjälper inte gruppen till stor del att genomföra dessa beslut, utan snarare endast att fatta dom. Det blir tydligt vad som ska göras men om ingen tydlig plan sätts upp så blir gruppen väldigt beroende på individuella insatser för att någon förändring ska synas. Därmed kan olika delar av gruppen ha väldigt olika uppfattning av hur gruppen fungerar tillsammans om det inte finns ett tydligt kollektivt samarbete kring potentiela förbättringar.

%This chapter contains a summarization of the purpose and the research
%questions. To what extent has the aim been achieved, and what are the
%answers to the research questions?

%The consequences for the target audience (and possibly for researchers
%and practitioners) must also be described. There should be a section
%on future work where ideas for continued work are described. If the
%conclusion chapter contains such a section, the ideas described
%therein must be concrete and well thought through.
