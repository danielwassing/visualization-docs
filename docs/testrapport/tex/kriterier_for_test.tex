\section{Kriterier för godkänd/misslyckat test}
För att ett test ska anses vara godkänt så måste det uppfylla testpåståendet enligt nedanstående standard för hur testfall och testpåståenden formuleras.
\subsection*{Testfall}
\textbf{TestID:} \textit{Unik identifierare. Varje testfalls ID är på formen Tx där x är ett löpnummer börjandes på 1.}\\
\textbf{Påstående:} \textit{Vad som ska verifieras. Se nedan (Formulering av testpåstående).}\\
\textbf{Antaganden:} \textit{De antaganden som understödjer testfallet.}\\
\textbf{Testdata: } \textit{Variabler och värden som testas.}\\
\textbf{Teststeg: } \textit{Arbetsgång för testet.}\\
\textbf{Förväntat resultat: } \\
\textbf{Uppnått resultat: } \textit{Vad resulterade testet i.}\\
\textbf{Godkänt/Underkänt test:} \textit{Huruvida testet godkändes eller ej.}\\
\textbf{Kommentarer:} 
\subsection*{Testpåstående}
För att formulera vad som ska verifieras används följande format: \\
\\
\textbf{Verifiera ...} \\
\textbf{... genom att använda ...} [verktyg, dialog] \\
\textbf{... med ...} [förutsättningar] \\
\textbf{... att ...} [vad som returneras, visas eller demonstreras]
\\ \\
För att testet ska betraktas som godkänt ska testet uppfylla följande kriterier:
\begin{itemize}
\item Testpåståendet är uppfyllt
\item Inga ytterligare antaganden är gjorda
\item Testdata har hämtats in
\item Teststegen har följts vid utförandet av testet
\item Det förväntade resultet har uppnåtts eller överträffats
\end{itemize}

\subsection*{Exempel}
\textbf{TestID:} T01\\
\textbf{Påstående:} Verifiera genom tidtagning med ramverkets tidtagninsfunktionalitet att hemsidan läses in på under två millisekunder. \\
\textbf{Antaganden:} Webbläsarens cacheminne är rensat. \\
\textbf{Testdata:} Uppmätt tid \\
\textbf{Teststeg:} Starta tidtagning, läs in hemsidan, läs av inläsningstiden. \\
\textbf{Förväntat resultat: } Hemsidans inläsningstid understiger två millisekunder. \\
\textbf{Uppnått resultat:} Hemsidan inläsningstid var en millisekund.\\
\textbf{Godkänd/Underkänt test: } GODKÄNT \\
\textbf{Kommentarer:} Testet är allmängiltigt för alla marknadsledande webbläsare. 