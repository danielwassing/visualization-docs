\section{Granskningar}
De granskningar som kommer att göras är dokumentgranskningar och kodgranskningar.
\subsection{Dokumentgranskningar}
För granskningar och korrekturläsning av dokument som ska levereras, enligt Tabell \ref{tab:tabell1}, finns det ett externt excelark med alla rubriker och delar för ett dokument där man skriver upp sig som ansvarig för det dokumentet. När rubriken är färdigskriven markerar man den rubriken i excelarket att det är redo för granskning, då kan en annan gruppmedlem skriva upp sig som granskningsansvarig för det dokumentet. Den ansvariga markerar sedan den rubriken som godkänd eller att korrigeringar behöver göras.
\subsection{Kodgranskningar}
Under projektets gång kommer lite olika sorters granskningar att användas i olika utsträckning. De som ska användas är parprogrammering, över-axelngranskning och mötesgranskning. Parprogrammeringen och över-axelngranskningen är inte så formella, medans mötesgranskningen är väldigt formell.
\subsubsection{Parprogrammering}
Under parprogrammering sitter man i par och programmerar på samma arbetsstation. Då kan den ena koncentrera sig på att skriva kod, medan den andra observerar och granskar koden samtidigt som den skrivs. Detta ger två synvinklar på koden och den som observerar kan tillföra information som den som kodar kan ha missat eller ge en ny synvinkel på ett problem.
\subsubsection{Över-axelngranskning}
Över-axelngranskning görs när koden är färdigskriven. Vid tillfället går den som skrivit koden högt igenom den, samtidigt står en person bakom eller sitter bredvid. Detta göra att den som skrivit koden kan hitta något som den inte tänkt på tidigare och observatören kan tillföra en ny synvinkel och komma med kritik eller annan input.
\subsubsection{Mötesgranskning}
Vid en mötesgranskning sätter sig en större del av projektgruppen i ett rum med projektor och utskrivna kopior av koden och går igenom den tillsammans. Då får man ett stort antal synvinklar på koden och alla kan ha lite olika saker som de värderar vid en granskningar. Detta tar upp mer resurser men ger en mer genomgående granskning av koden.