\section{Organisationsplan för projektet}
Här presenteras projektets organisation.

\subsection{Roller}
De olika roller som projektgruppen valde att arbeta i var:\\
\begin{itemize}
\item Teamledare - Johan Nåtoft
	\begin{itemize}
		\item Teamledarens roll är att se till att projektets mål uppfylls. Det är även dennes ansvar att vara projektgruppens ansikte utåt och sköta all kontakt med projektledningen. Teamledaren är också en coach till medlemmarna i projektet och ska agera bollplank samt ska se till att gruppen har ett gott samarbete sinsemellan. 
	\end{itemize}
\item Konfigurationsansvarig - Jonathan Wahlund
	\begin{itemize}
		\item Bestämmer vilka produkter som skall versionshanteras samt vilka arbetsprodukter som skall ingå i en utgåva. Ser till att verktygen som används för versionshantering används på rätt sätt och att dessa är underhållna. Är även ansvarig för att hitta lösningar till problem som gruppen kan stöta på med verktygen. 
	\end{itemize}
\item Testledare - Joakim Argillander
	\begin{itemize}
		\item Beslutar om systemets status och ansvarar för att en testplan samt testrapport skrivs. Ansvarig för att tester utformas enligt rådande och överenskommen standard. Ansvarig för att kravspecifikationen är testbar och att testfall körs för alla testbara krav. Delegerar testuppgifter och sätter upp en testplan i mån av uppgifter och tidsplan för denna.
	\end{itemize}
\item Kvalitetsansvarig - Victor Bodin
	\begin{itemize}
		\item Ansvarig för att sammanställa en kvalitetsplan, samla in erfarenheter från gruppens medlemmar och dokumentera dessa. Planerar även för eventuell utbildning och inspiration för gruppen. Samarbetar med samtliga av gruppens medlemmar för att nå en hög nivå av kvalitet genom hela projektet. Ser även till att följa upp på de aktiviteter som vi beslutar om så att vi vet vad som kan göras bättre eller tydliggöra vad vi gjorde för fel.
	\end{itemize}
\item Utvecklingsledare - Sebastian Callh
	\begin{itemize}
		\item Ansvarig för fördelningen av utvecklingsarbetet och den detaljerade designen på systemet. Gör beslut angående utvecklingsmiljö och är även Scrum-master, där uppgifter så som att organisera utvecklingsarbetet och bistå med teknisk kompetens ingår.
	\end{itemize}
\item Arkitekt - Johan Thornström
	\begin{itemize}
		\item Ansvarig för att ta fram en arkitektur och identifiera komponenter samt gränssnitt. Gör även de övergripande teknikvalen och har även mest inflytande i tekniska frågor. Samrådar med utvecklingsledaren hur tekniska frågor ska tacklas och ansvarar för att gränssnitt finns och följs.
	\end{itemize}
\item Analysansvarig - Rebecca Lindblom
	\begin{itemize}
		\item Denna person sköter kontakten med kund och är även direkt ansvarig för kraven som ställs på projektet samt för att dokumentera dessa. Analysansvarig ansvarar även för att kravspecifikationen blir skriven även om personen inte är ensam om att skriva den.
	\end{itemize}
\bigskip
\item Dokumentansvarig - Daniel Wassing
	\begin{itemize}
		\item Ansvarig för skapandet av dokumentmallar och struktur. Även ansvarig för projektetslogga samt har en övergripande koll på deadlines för olika dokument. Är även ansvarig för att dokumentera på möten som hålls och att revidera dessa så att de är tydliga och användbara.
	\end{itemize}
\end{itemize}

\subsection{Kunskap}
Projektgruppens kunskap är varierande och bred. Gruppen ska jobba kontinuerligt för att dela med sig av nödvändig kunskap till samtliga medlemmar.
Den kunskap som förväntas att finnas hos gruppmedlemmarna är de förkunskapskrav som ställs på alla studenter som gör kandidatarbetet.

\subsection{Utbildning}
Projektgruppen kommer ta del av interna utbildningar inom testning, Git och JavaScript i största möjliga mån och vid behov. Vi värdesätter att vår kod är bra så vi har haft en utbildning inom testbar kod.
\begin{itemize}
\item Powerpoint slides för testbar kod - \url{https://goo.gl/oUiBwL}
%lägg till mer utbildningsgrejer här
\end{itemize}

\subsection{Kommunikation}
All kommunikation inom gruppen som inte sker fysiskt sker genom verktyget Slack. Kommunikation med kunden går genom teamledaren och analysansvarig via mail.

\subsection{Rapporter}
Intern dokumentering och organisation sker på Google Drive medan alla officiella dokument produceras i LaTeX och versionhanteras genom Git. I tabell \ref{tab:dokumentation} nämns alla officiella dokument som kommer produceras under projektets gång. Alla gruppmedlemmar rapporterar den tid de arbetat med projektrelaterade uppgifter i ett rapporterings-dokument på Google Drive. 

\subsection{Agil arbetsmetodik}
Projektets utveckling kommer att bedrivas enligt Scrum, med diverse modifikationer för att fungera tillsammans med gruppens övriga studier. En jämförelse mellan renordnad Scrum och vår tillämpning av det finns i appendix A.