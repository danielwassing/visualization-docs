\chapter{\hspace{2.6em} Frågeformulär för utvärdering av workshop}

\section{Frågeformulär inför workshop}
\label{sec:workshop_undersokning}

\begin{table}[h!]
  \centering
  \caption{En tabell över frågorna i frågeformuläret inför workshoppen.}
  \def\arraystretch{1.5}
  \begin{adjustbox}{max width=\textwidth}
    \begin{tabularx}{\textwidth}{ | l | X | l |}
      \hline
      \textbf{Nummer} & \textbf{Fråga} & \textbf{Svarstyp} \\
      \hline
      1 & Jag känner att jag har de färdigheter som projektet kräver av mig & Skala 1-5 \\
      \hline
      2 & Det finns färdigheter relaterade till projektet som jag kan förbättras inom & Skala 1-5 \\
      \hline 
      5 & Om ja, vilka färdigheter? & Text \\
      \hline
      3 & Vilka färdigheter relaterade till projektet är viktigast enligt dig? & Text \\
      \hline
      4 & Jag känner att jag i praktiken har koll på hur automatiserade tester skrivs i projektet & Skala 1-5 \\
      \hline
      6 & Finns det några övriga färdigheter du skulle vilja förbättras inom? & Text \\
      \hline
    \end{tabularx}
  \end{adjustbox}
  \label{tab:workshop_undersokning}
\end{table}

\pagebreak
\section{Frågeformulär efter workshop}
\label{sec:workshop_efter_formular}

\begin{table}[h!]
  \centering
  \caption{En tabell över frågorna i frågeformuläret efter workshop.}
  \def\arraystretch{1.5}
  \begin{adjustbox}{max width=\textwidth}
    \begin{tabularx}{\textwidth}{ | l | X | l |}
      \hline
      \textbf{Nummer} & \textbf{Fråga} & \textbf{Svarstyp} \\
      \hline
      1 & Workshoppens innehåll upplevdes relevant för projektet & Skala 1-5\\
      \hline
      2 & Vilka moment, om några, upplevdes relevanta? & Text \\
      \hline
      5 & Workshoppens svårighetsgrad vad lagom? & Skala 1-5 \\
      \hline
      3 & Vad vad bra? & Text \\
      \hline
      4 & Vad kunde gjorts bättre? & Text \\
      \hline
      6 & Användandet av Powerpoint bidrog till förståelse & Skala 1-5 \\
      \hline
      7 & Användandet av Live-kodning bidrog till förståelse & Skala 1-5 \\
      \hline
    \end{tabularx}
  \end{adjustbox}
  \label{tab:workshop_efter_formular}
\end{table}

\section{Frågeformulär efter utveckling}
\label{sec:workshop_formular_efter_utveckling}

\begin{table}[h!]
  \centering
  \caption{En tabell över frågorna i frågeformuläret efter utveckling.}
  \def\arraystretch{1.5}
  \begin{adjustbox}{max width=\textwidth}
    \begin{tabularx}{\textwidth}{ | l | X | l |}
      \hline
      \textbf{Nummer} & \textbf{Fråga} & \textbf{Svarstyp} \\
      \hline
      1 & Workshoppens innehåll har varit användbart inom projektet & Skala 1-5\\
      \hline
      2 & Vilka bitar från workshoppen har du använt dig av?? & Flerval \\
      \hline
      5 & Om det finns moment du inte använd dig av, varför? & Text \\
      \hline
      3 & Vilka moment saknades? & Text \\
      \hline
      4 & Upplever du att workshoppen höjt din produktivitet? & Ja/Nej \\
      \hline
      6 & Om inte, hur kunde det ha uppnåtts? & Text \\
      \hline
    \end{tabularx}
  \end{adjustbox}
  \label{tab:workshop_formular_efter_utveckling}
\end{table}

\chapter{\hspace{2.6em} Kravspecifikation för system till Donaldknuthias regering}
\label{sec:workshop_kravspec_architecture}
\section{Arkitekturkrav}

\begin{table}[h!]
  \centering
  \caption{En tabell över systemets arkitekturkrav.}
  \def\arraystretch{1.5}
  \begin{adjustbox}{max width=\textwidth}
    \begin{tabularx}{\textwidth}{ | l | X | }
      \hline
      \textbf{Krav} & \textbf{Beskrivning} \\
      \hline
      KA1 & Systemet skall vara byggt i Meteor \\
      \hline
      KA2 & Systemet skall använda automatiserade tester \\
      \hline
    \end{tabularx}
  \end{adjustbox}
  \label{tab:workshop_arkitekturkrav}
\end{table}

\pagebreak

\section{Funktionella krav}
\label{sec:workshop_kravspec_functional}

\begin{table}[h!]
  \centering
  \caption{En tabell över systemets funktionella krav.}
  \def\arraystretch{1.5}
  \begin{adjustbox}{max width=\textwidth}
    \begin{tabularx}{\textwidth}{ | l | X | }
      \hline
      \textbf{Krav} & \textbf{Beskrivning} \\
      \hline
      KF1 & Användare ska ha en säkerhetsnivå \\
      \hline
      KF2 & Alla användare ska kunna göra sin skattedeklaration \\
      \hline
      KF3 & Systemet skall använda automatiserade tester \\
      \hline
      KF4 & Användare som innehar säkerhetsnivå 2 skall få avfyra missilerna \\
      \hline
      KF5 & Användare som inte har säkerhetsnivå 2 skall ej få avfyra missilerna \\
      \hline
    \end{tabularx}
  \end{adjustbox}
  \label{tab:workshop_funktionellakrav}
\end{table}
