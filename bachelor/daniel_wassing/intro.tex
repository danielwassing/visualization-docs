%%% Intro.tex --- 
%% 
%% Filename: Intro.tex
%% Description: 
%% Author: Ola Leifler
%% Maintainer: 
%% Created: Thu Oct 14 12:54:47 2010 (CEST)
%% Version: $Id$
%% Version: 
%% Last-Updated: Thu May 19 14:12:31 2016 (+0200)
%%           By: Ola Leifler
%%     Update #: 5
%% URL: 
%% Keywords: 
%% Compatibility: 
%% 
%%%%%%%%%%%%%%%%%%%%%%%%%%%%%%%%%%%%%%%%%%%%%%%%%%%%%%%%%%%%%%%%%%%%%%
%% 
%%% Commentary: 
%% 
%% 
%% 
%%%%%%%%%%%%%%%%%%%%%%%%%%%%%%%%%%%%%%%%%%%%%%%%%%%%%%%%%%%%%%%%%%%%%%
%% 
%%% Change log:
%% 
%% 
%% RCS $Log$
%%%%%%%%%%%%%%%%%%%%%%%%%%%%%%%%%%%%%%%%%%%%%%%%%%%%%%%%%%%%%%%%%%%%%%
%% 
%%% Code:


\chapter{Introduktion}
\label{cha:wassing-introduction}
Projektdokumentation är en kritisk roll för lyckade projekt. Dokumentation behövs i många stadier och alla är högt relevanta för projektets framgång. Från att en projektplan och kravspecifikation tas fram och skrivs tills det att en slutrapport med användarhandledning lämnas in så spelar dokumentationen en central roll i projektets arbetsgång och tidsram. 
\\ \\
Hur man skriver dokument lär man sig redan från barnsben och man slutar aldrig lära sig - på universitetet får man fortfarande lära sig hur man skriver akademiskt korrekt. Men när det gäller verktygen för dokumentation så erbjuds det i dagsläget ingen kurs på Linköpings Universitet för hur man använder dem, och verktygen spelar en väldigt stor roll i hur framgångsrik dokumentationen är.

\section{Motivation}
\label{sec:wassing-motivation}
Allt eftersom mjukvaruindustrin växer så växer även behovet av att dokumentera korrekt och effektivt. Då det kan ta väldigt mycket tid att dokumentera så är det väldigt viktigt att det effektiviseras. Olika projektmedlemmar kan behöva hjälpa till med dokumentation, och då är det viktigt att alla kan jobba parallellt utan att olika data går förlorade eller skrivs över. Detta kan uppnås på flera sätt, men då studenter på utbildningen för datateknik inte får någon utbildning inom olika dokumentationsmetoder så är det svårt, om inte omöjligt, att avgöra vad man bör använda för verktyg för att vara så tidseffektiv som möjligt vid dokumentation av projekt. Olika verktyg medför olika för- och nackdelar samt utmaningar.

\section{Målsättning}
\label{sec:wassing-aim}
Den här rapporten ämnar att tydliggöra vilka för- och nackdelar olika verktyg medför när man dokumenterar för ett projekt, dvs. vad man kan spara tid på och vilka fördelar man kan få i slutdokumentationen. Rapporten ämnar även att belysa vilka utmaningar grupper kan ställas inför när de väljer hur de ska dokumentera sina projekt och vad de skulle kunna basera sina val på.

\section{Frågeställningar}
\label{sec:wassing-research-questions}
Den största utmaningen för valet av verktyg i projektdokumentation för kandidatprojektet i programvaruutveckling är i dagsläget gruppens kunskap, då ingen vet vad som är tidseffektivast. Nya verktyg kan ta väldigt mycket tid att lära sig, och det är ett stort steg från att ha fått grepp om deras grundläggande funktioner och attribut tills det att man arbetar effektivt med dem. Studiens frågeställning är således: \\

\textbf{"Val av verktyg för dokumentation, vilka kriterier är viktiga?"} \\ \\
För att bryta upp ovanstående frågeställning så kommer två aspekter besvaras noggrannare:
\begin{enumerate}
\item \textit{Vilka verktyg finns tillgängliga för att dokumentera i ett projektarbete och vilka uppenbara för- och nackdelar medför de?}
\item \textit{Vilka kriterier väger tyngst vid valet av verktyg att dokumentera med? Kan användandet av flera verktyg kombineras och vilka verktyg lämpar sig bäst i så fall?}
\end{enumerate}

\section{Avgränsningar}
\label{sec:wassing-delimitations}
Denna rapport kommer inte ta hänsyn till de olika gruppernas projekt, även om det finns möjlighet att projektens karaktär och egna avgränsningar kan ha bidragit till resultatet. Det kommer inte heller tas hänsyn till medlemmars tidigare erfarenheter, även om vissa medlemmar kan ha erfarenheter som kan påverka resultatet i denna rapport. \\
Studien i sin helhet baseras på kandidatarbetet som utförs av grupp 2 i kursen TDDD96 våren 2017, interjvuer som hållits med några medlemmar ur slumpvis utvalda grupper, samt de referenser som nämns i det här dokumentet.

%\nocite{scigen}
%We have included Paper \ref{art:scigen}

%%%%%%%%%%%%%%%%%%%%%%%%%%%%%%%%%%%%%%%%%%%%%%%%%%%%%%%%%%%%%%%%%%%%%%
%%% Intro.tex ends here


%%% Local Variables: 
%%% mode: latex
%%% TeX-master: "demothesis"
%%% End: 
