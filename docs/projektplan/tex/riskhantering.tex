\section{Riskhantering}
\subsection{Risker} 
Följande risker har identifierats som möjliga under projektets gång. Sannolikheten för att risken ska inträffa har bedömts 1-5, och en värdering (1-5) av vilken påverkan det har för projektet har gjorts. Produkten av dessa bildar en riskmagnitud som graderar riskfallen.
\\
\\
\textbf{Risk: } (1) Projekets storlek underskattas och projektet riskerar därför att inte kunna levereras i tid. \\
\textbf{Sannolikhet: } 2	\\
\textbf{Påverkan: }	5		\\
\textbf{Riskmagnitud: } 10	\\
\textbf{Hantering:} Projektet utförs i korta iterationer och tidiga prototyper demonstreras för kund. 
\\
\\
\textbf{Risk: } (2) Projektmedlemmar är sjuka en längre tid eller slutar i projektet. \\
\textbf{Sannolikhet: } 1	\\
\textbf{Påverkan: }	4		\\
\textbf{Riskmagnitud: } 4	\\
\textbf{Hantering:} Tillsätt vice-roller för varje ansvarsområde, och se till att varje gruppmedlem ser till att hålla sin rolls dokument uppdaterat. 
\\
\\
\textbf{Risk: } (3) De givna ramverken och plattformarna som är givna i kravspecifikationen når inte prestandakraven. \\
\textbf{Sannolikhet: } 4	\\
\textbf{Påverkan: }	5		\\
\textbf{Riskmagnitud: } 20	\\
\textbf{Hantering:} Avgränsa projektresultatet till det som är möjligt utifrån de givna ramverken. Utveckla tidiga prototyper för att säkerställa att tekniken fungerar. \\
\\
\textbf{Risk: } (4) De givna ramverken och plattformarna som är givna i kravspecifikationen visar sig ha en hög inlärningströskel som hindrar utveckling.\\
\textbf{Sannolikhet: } 4	\\
\textbf{Påverkan: }	4		\\
\textbf{Riskmagnitud: } 16	\\
\textbf{Hantering:} Arbeta med internutbildning och avsätt tid för inläsning och experimentering. 